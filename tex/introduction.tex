\chapter{\color{burntorange}{Introduction}}

	\section{Contexte}

		\paragraph{Développement logiciel \\}
		
			Le contexte de cette thèse se focalise sur le monde et l'environnement de développement logiciel au coeur de l'informatique. L'\Gls{open source} aujourd'hui, peut s'étendre aux oeuvre de l'esprit dont il ne sera pas traité particulièrement mais désignera les logiciels dit "ouverts".

		\paragraph{Sous licence\\}

			Une partie du contenu de cette thèse a été inspiré d'un ouvrage sous licence \gls{Creative Commons} "Paternité-pas de Modifications" 2.0. 
			Les modifications apportés à l'oeuvre originale rendent donc ce document non distribuable.
		
		\paragraph{Introduction à l'open source\\}

			Afin de lever l'ambiguïté entre les logiciels "libres" (en anglais "free", pouvant désigner un logiciel gratuit, ou libre) et les logiciels "ouverts", l'expression "\gls{open source}" est apparue en 1998 par Christine \bsc{Peterson} du Foresight Institute.
			Richard \bsc{Matthew Stallman} défend le terme de \textit{free software} à travers son organisme la \acrfull{fsf}
			Eric \bsc{Raymond} et Bruce \bsc{Perens} créent en 1998, l'\acrfull{osi}, qui délivre le label "\acrshort{osi} approved" aux licences qui satisfont aux critères définis dans l'Open Source Definition.

			\begin{center}
				\textit{
				\textquote{
					L'Open Source permet une méthode de développement de logiciels qui exploite la puissance de l'évaluation par les pairs distribuée et la transparence des processus. La promesse de l'open source est une meilleure qualité, une meilleure fiabilité, une plus grande flexibilité, des coûts moins élevés et la fin de l'immobilisme prédateur des fournisseurs.} - \acrshort{osi}
				}
			\end{center}

			Le code source est la version d'un programme qui est lisible et intelligible pour l'homme. C'est le code source qui est écrit par l'informaticien, le programmeur, et qui pourra être relu et modifié par d'autres. Les programmes peuvent ensuite être compilés, ce qui produit le code \textit{objet}, ou \textit{binaire}, ou encore \textit{exécutable}, qui lui n'est pas compréhensible.

			Un logiciel \textit{libre}, ou logiciel \textit{open source}, est un programme dont le code source est distribué et peut être utilisé, copié, étudié, modifié et redistribué sans restriction.

			Un logiciel dit open source ou logiciel libre ne signifie pas gratuit.
			Rien n'interdit de faire payer la distribution d'un logiciel bien que celui à qui il sera vendu pourra le redistribuer gratuitement.
			Dans la pratique il est donc considéré qu'un logiciel open source est généralement gratuit. Nous verrons ultérieurement, les différents moyens de générer de l'argent avec de l'open source par le biais d'intégration, support, formation, développement supplémentaire.

			Les deux appellations «open source» et «logiciel libre» sont presque équivalentes, mais correspondent à des écoles de pensées différentes.

			La commission européenne adopte le terme \acrfull{floss}. J'ai pris le parti d'utiliser le terme \textit{open source} pour simplifier la lecture.
				
		\paragraph{Pourquoi l'open source ?\\}

			En 2019, une étude de la société RedHat, leader dans le monde de l'open source, menée sur 950 entreprises rapporte que 89\% d'entres elles  considèrent l'open source comme important si ce n'est plus.

			En tant que développeur j'ai découvert l'open source au sein de mon entreprise actuelle, Docdoku.
			Docdoku développe un logiciel open source aidant de nombreuses entreprises.
			Je constate également qu'il existe encore beaucoup d'amalgames autour de l'open source, des défauts et abus de langage.\\
			Les institutions et entreprises dans le monde de l'informatique ne sensibilisent pas assez les développeurs actuels et en devenir sur l'open source. Les avantages qu'offrent celui-ci,l'aspect juridique, les conditions d'utilisation et bien d'autres caractéristiques propres de l'open source ne sont pas enseignés.\\

			Je pars donc du principe que l'open source ne fait pas assez de "bruit" dans le monde, ainsi je souhaite développer à travers cette thèse, l'intrigation, la passion et l'envie de nous tourner vers l'open source, d'y contribuer et d'en faire la solution privilégiée des développeurs et des consommateurs occasionnels.\\

			Voici deux affirmations que beaucoup d'entre nous ont faites ou peuvent faire.
			
			\subparagraph{« L'open source c'est moins bien »\\}

				L'open source est souvent associé à de mauvais termes. On peut entendre logiciel gratuit, libre, non supporté par une entreprise, développé par le particulier ou des communautés en tant que passe-temps. Un manque de rigueur peux donc s'en dégager et dévaloriser l'image de celui-ci auprès des éditeurs et consommateurs.

			\subparagraph{« L'open source n'a pas besoin de moi »\\}

				L'open source ne se résume pas seulement au développement de code informatique et est un vaste domaine ou chaque individu, quelque soit ses compétences, peut y participer afin de rendre le monde et l'informatique meilleure qu'elle n'est.

	\section{Problématique}
		\paragraph*{}

			Je vais donc vous présenter la problématique suivante :
			\begin{center}
				\begin{displayquote}
					\textbf{Comment valoriser, en tant qu'éditeur, l'open source et en faire la solution privilégiée des consommateurs ?}
				\end{displayquote}
			\end{center}

		\paragraph*{}
			
			Afin de répondre à cette problématique, je vais m'appuyer sur des hypothèses fondées sur mon vécu en entreprise mais également issues de ma réflexion personnelle dans mon métier de développeur logiciel.

	\section{Hypothèses}

		\paragraph{Les plateformes promotrices\\}

			Selon moi si le nombre de plateforme hébergeant du contenu open source et offrant la possibilité d'y contribuer mettaient plus en avant le contenu et les projets, l'open source serait plus prisé par les consommateurs et l'on constaterait plus de contributions à ces projets.

		\paragraph{L'optimisation\\}

			Si nous améliorons l'aspect communautaire dans son organisation,donnons accès à des formations, inclure et impliquer un peu mieux les personnes sans pour autant compliquer la participation à celle-ci, nous valoriserons l'open source.

		\paragraph{Les envies et besoin des consommateurs\\}

			Si les consommateurs possédait des moyens efficaces afin d'exprimer leur souhaits et besoins en terme d'open source, ils en seraient plus satisfait et cela en ferait la solution idéale à leurs yeux.

	\section{Méthodologie de travail / périmètre}
		\paragraph{Etude de l'open source}

			\subparagraph{Présentation de l'open source\\}

				Afin d'avoir un retour sur mes hypothèses, il m'est nécessaire d'avoir une meilleure vision d'ensemble de l'open source et un périmètre défini sur son impact, les conditions, les chiffres et son implication dans le monde.

			\subparagraph{Les éditeurs open source\\}

				Je me positionne du côté des éditeurs, pour diverse raison évoqués précédemment, je souhaite donc en savoir plus sur l'éditeur, ses rôles, son périmètre d'action.

			\subparagraph{L'optimisation des ressources\\}
		
				Après avoir fait la lecture et l'analyse personnelle de plusieurs ouvrages dont 2 principaux sur lesquels je m'appuyerai : 
		
				\begin{displayquote}
					\textquote{Réfléchissez et devenez riche} - Napoleon \bsc{Hill} - 1937
				\end{displayquote}
				\begin{displayquote}
					\textquote{Oser la confiance} - Bertrand \bsc{Martin} (†), Vincent \bsc{Lenhardt}, Bruno \bsc{Jarrosson} - 1996
				\end{displayquote}
		
				Je souhaite effectuer des recherches sur l'optimisation (et ses moyens) des ressources qui définissent l'open source.

			\subparagraph{Le consommateur\\}
				En étudiant le consommateur, nous verrons les besoins et activités relatives à l'open source.

			\subparagraph{Le marché de l'open source\\}
				Enfin nous verrons l'aspect économique associé à l'open source en étudiant son marché.

		\paragraph{Sur le terrain \\}
			Lors de mon étude terrain je consoliderai mes recherches dans les domaines cités précédemment pour pouvoir les confronter.

		\paragraph{L'analyse et les résultats \\}
			Je transposerai alors mes études et relevés afin de valider ou réfuter mes hypothèses et apporter une réponse acceptable à ma problématique en guise de conclusion.