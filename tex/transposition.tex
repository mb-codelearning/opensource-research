\chapter{Transposition} % 3/4 pages

Afin de répondre à ma problématique suivante:

\textbf{Comment valoriser, en tant qu'éditeur, l'open source et en faire la solution privilégiée des consommateurs ?}

Je vous apporte donc des préconisations à mettre en place autour de différents domaines.

\section{Améliorer la communication}

	Autour de ce travail, j'ai pu dégager l'évidence d'un besoin de communication vital pour le bien être du consommateur de l'open source mais également de l'éditeur, ainsi je préconise aux éditeurs de logiciel open source d'accentuer leur documentations pour une meilleure communication.

	\paragraph{Documentation inhérente au projet\\}

	Autour de chaque projets hébérgés sur les plateformes, une page d'accueil est généralement disponible. Le "ReadMe" est le fichier qui est présenté dans cette page et c'est celui qui doit être le plus complet pour appeler à la contribution.\\

	Il doit ainsi contenir les informations essentielles pour la contribution. Comment s'y prendre, quelles sont les étapes pour cela, les prérequis, les étapes de validations tout en restant le plus simple possible. Il existe même une méthode de développement autour de ce ReadMe, le \emph{Readme Driven Development}, consistant à mettre la priorité sur le Readme avant tout développement.

	\paragraph{Utilisation d'outils de communication\\}

	Pour développer en équipe, il existe de nombreux outils facilitant la communication. Mettre à disposition un espace de partages et d'échanges comme le propose Slack ou Reddit permettra de gérer les communications au travers de la communauté et de faire ressortir les idées et besoins.

	\paragraph{Intelligence collective\\}

	Cyber connecter l'intelligence collective des entreprises, des développeurs afin de bâtir les briques logicielles open source de demain. Avec la mise à disposition d'une interface non pas pour contribuer mais pour discuter et échanger sur les besoins de demain et les projets à batir.

\section{Sensibiliser à l'open source}

	Dans le but de promouvoir son propre produit open source, il peut être intéressant de sensibiliser les entreprises et développeur.

	\paragraph{Conférence sur l'intéret économique de l'open source\\}

	Participer aux différents évènements présents autour du monde du logiciel en apportant un témoignage de votre expérience de l'open source afin de solliciter les entreprises à contribuer en open sourçant leur produits également.

	Ceci apportera plus de consommateurs à l'open source et de contributions inter-entreprises.

	\paragraph{Accompagner à l'open source\\}

	Il peut être envisagé de se faire accompagner quand l'on souhaite open sourcer son code et continuer sur ce modèle, certaines connaissances et mise en place sont nécessaires. Ainsi proposer des services d'accompagnement aux futurs éditeurs permettra de croître le nombre de produits open sources sur le marché.

\section{Faciliter la contribution}
	
	\paragraph{Accompagner le contributeur\\}

	Le développeur qui souhaite contribuer à un projet open source se doit d'être accompagné dans sa démarche lors de ses premières contributions. Un accompagnement personnalisé est un plus qui peut amener facilement de nouveaux consommateurs à votre produit.\\

	La mise en place de didacticiels, d'échanges avec le contributeur permet d'améliorer la qualité des relations humaines dans le projet mais également de s'assurer de la bonne contribution du volontaire.

	\paragraph{Un guide de contribution\\}

	Pour faciliter la tâche à l'éditeur, un guide de contribution sur un format clair et rapide à lire est un atout essentiel.
	Il permet non seulement d'attirer la contribution mais également de gagner du temps d'échange, d'instruction à l'éditeur envers la communauté.\\

	Se référer à celui-ci permettra d'avoir un exemple de ligne de conduites à suivre pour participer à l'open source sans difficultés.