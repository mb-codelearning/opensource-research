\chapter*{Abstract}

\section*{English}
	
	\paragraph*{Introduction\\}

	This document aims to expose the different arguments on how to enhance the open source work as an editor and promote open source to be the main solution picked by potential consummers.

	Open source is very important in IT and moreover is a school of thought. Understanding how and why it matters will increase the number of contributions and help the IT's world. That is why I offer you a first step into open source from the editor's side.

	My hypotheses are that we can improve the promotion made to the open source from the hosting platform. We can optimize the human ressources that contribute to the project by attributing their work and specilizing them and also that we need to raise awareness of the world about the open source importance.

	\paragraph*{First part\\}

	I did my research, over all, about the open source and specificaly the aspect that can be improved in order to make it the best solution for its consummers.
	My studies gather informations about the open source's consummer: it expectations, it needs and requirements. You'll also find more about the editor and the business model that works with the open source project.

	In order to improve the open source community sharing work, I did research about the collective intelligence from books and the horizontal management of project.

	Lastly, I found some aspect on what kind of marketing could be done to promote the open source solutions.

	Once this research done, I started to ask people's opinion about my thought related to open source.

	\paragraph*{Second part\\}

	I did some interview with people concerned by open source. In addition a form was sent and treated by 40 persons.

	The answers reflect multiple improvements that can be done on the communication aspect in general, whether it is for the consummer's expectation that needs to be heard or the editor that should guide contributions.

	It has also enlightened my vision about the collective aspect and project management of the open source work. I understood that all contributions made to the open source are on a voluntary basis, hence the discrepency between my expectations to attribute specific work to contributors on the project and their free will.

	Make companies and developers aware of the importance of open source and its interests is a fact that emerged from this study.

	Finally, despite the marketing aspect coming from the "open" part of the open source, a classic marketing to promote the product is still required.

	I was now able to compare my opinion to people's one in order to answer my hypotheses and give an anwser for editors to enhance the open source.

	\paragraph*{Hypotheses answered\\}

	This confrontation revealed that my hypothesis about the awareness of people that need to be raised is correct but does concern more the companies and the developers rather than the entire world.

	I was wrong about the optimization of the human ressources by attributing their work using collective intelligence and managing them because contributors can not be managed.

	Hosting platforms can be improved, not in the way that I expected, but on the documentation aspect that will help people to contribute easily. So my hypothesis about that is partially valid

	\paragraph*{Conclusion\\}

	To begin your journey in open source as an editor, this study will show you some levers that should be considered such as communications, awareness of people and companies and guides for first contributors.

\section*{Français}

	\paragraph*{Introduction\\}

	Ce document a pour but d'exposer les différents arguments sur la façon d'améliorer l'open source en tant qu'éditeur et de promouvoir celui-ci afin d'en faire la solution principale pour les consommateurs.

	Je vous parle de l'open source car il s'agit d'un sujet très important pour l'informatique mais également d'un mouvement de pensée.
	En comprenant le pourquoi et le comment de son importance, cela augmentera peut-être le nombre total de contributions et améliorera le monde de l'informatique. C'est pourquoi je vous parle de l'open source.

	Mon hypothèse est que nous pouvons améliorer la promotion faite à l'open source à partir de la plateforme d'hébergement. Nous pouvons optimiser les ressources humaines qui contribuent au projet en attribuant leur travail et en les spécialisant. Egalement, je pense que nous avons besoin de sensibiliser le monde à l'importance de l'open source.

	\paragraph*{Première partie\\}

	J'ai fais mes recherches sur l'open source et plus particulièrement sur ce qui peut être amélioré pour en faire la meilleure solution aux yeux des consommateurs.

	Mes études rassemblent des informations sur le consommateur de l'open source : ses attentes, ses besoins et ses exigences. Vous y trouverez des informations sur l'éditeur et le modèle économique qui fonctionne avec le projet open source.

	Afin d'améliorer le travail de partage de la communauté open source, j'ai fait des recherches sur l'intelligence collective issue d'ouvrages et la gestion horizontale de projet.

	Enfin, j'ai trouvé un aspect sur le type de marketing qui pourrait être fait pour promouvoir les solutions open sources.

	Une fois cette recherche effectuée, j'ai commencé à demander aux gens leur opinion sur le sujet.

	\paragraph*{Deuxième partie\\}

	J'ai fait quelques interviews avec des personnes concernées par l'open source. De plus, un formulaire a été envoyé et traité par 40 personnes.

	Les réponses reflètent les multiples améliorations qui peuvent être apportées sur l'aspect communication en général, que ce soit pour les attentes du consommateur qui doivent être entendues ou de l'éditeur qui doit guider les contributions.

	Cela a également éclairé ma vision sur l'aspect collectif et la gestion de projet du travail open source. J'ai compris que toutes les contributions faites à l'open source le sont sur une base volontaire, d'où le décalage entre mes attentes d'attribuer un travail spécifique aux contributeurs sur le projet et leur libre-arbitre.

	Faire prendre conscience aux entreprises et aux développeurs de l'importance de l'open source et de ses intérêts est un fait qui ressort de cette étude.

	Enfin, malgré l'aspect marketing venant de la partie "open" de l'open source, un marketing classique pour promouvoir le produit open source est encore nécessaire.

	J'ai donc pu comparer mon opinion à celle des gens afin de répondre à mes hypothèses et de donner une réponse aux éditeurs pour améliorer l'open source.

	\paragraph*{Réponses aux hypothèses\\}

	Cette confrontation a révélé que mon hypothèse sur la prise de conscience des personnes à sensibiliser est correcte, mais qu'elle concerne plus les entreprises et les développeurs que le monde entier.

	Je me suis trompé sur l'optimisation des ressources humaines en pensant attribuer leur travail par l'intelligence collective, car les contributeurs ne peuvent pas être managés.

	Les plateformes d'hébergement peuvent être améliorées, pas de la manière que j'attendais, mais plus sur l'aspect documentation qui aidera les gens à contribuer facilement. Mon hypothèse à ce sujet est donc partiellement valable.

	\paragraph*{Conclusion\\}

	Pour débuter votre parcours dans l'open source en tant qu'éditeur, cette étude vous montrera quelques leviers à considérer tels que la communication, la sensibilisation des personnes et des entreprises et des guides pour les premiers contributeurs.





