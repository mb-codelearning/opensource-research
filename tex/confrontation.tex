\chapter{Confrontation} % 3 pages

% Confronter etat de l'art et etude terrain et donc Réponse aux hypothèse en disant validé à 80\% ...

\section{Promotion de l'open source}

	Ma première hypothèse concerne le manque de promotion, de mise en avant des produits open source, et du manque d'outil disponible sur les plateforme d'hébergement de code open source pour en faire le produit phare de tout les développeurs et entreprises du logiciel.

	\subsection{Des plateformes améliorables}

	Dans mon étude autour de l'open source, je me suis aperçu de la multitude des plateformes disponibles pour présenter les projets open sources et permettre la contribution. Malgré celà, il me semblait nécessaire de mettre en place une interface pour accueillir les potentiels futur contributeur, une vitrine permettant de visualiser le produit.

	Avec l'étude terrain et les résultat à mes questions autour de ce sujet, je me suis aperçu que les plateformes n'ont pas nécessité à promouvoir le projet car l'aspiration à la contribution n'est pas lié à la motivation et au marketing qui gravite autour. C'est sur besoin qu'a le consommateur à utiliser une spécificité du logiciel open source, et donc à communiquer son idée, ses besoins pour  et orienter le développement du produit vers son besoin.

	Ainsi mon hypothèse sur l'interface de promotion des projets open source n'est pas valide.

	Néanmoins il s'en est dégagé un besoin crucial d'améliorer la communication entre toutes les parties prenantes du projet open source.

	\subsection{Le marketing de l'open source}



\paragraph{Réponse à l'hypothèse}

\section{Optimisation des ressources}

	\subsection{Business model de l'open source}

	\subsection{Gestion des ressource humaines}

\paragraph{Réponse à l'hypothèse}

\section{Envies et besoins de contribuer}

	Ma troisième hypothèse traite du manque de sensibilisation qui plane sur l'open source, faisant de celui-ci un sujet de malentendu, d'incompréhension ou d'ignorance par les consommateurs potentiel. Si l'on sensibilisait plus le consommateur de l'open source sur l'importance de leur contribution, alors le blason de l'open source en serait redoré et attirerait encore plus de monde.

\subsection{Sensibiliser le grand public}

	Lors de mes recherches autour de l'open source et en comparant avec mon vécu dans mes entreprises et à l'école, je me suis aperçu du manque de sensibilisation sur le vaste sujet qu'est l'open source.\\

	Il est vrai que je n'ai que vaguement entendu ce sujet à quelques reprises avant d'en faire mon sujet d'étude.
	J'ai donc profité de mon étude terrain pour aller questionner d'autres personnes sur ce sentiment de manque d'information autour de l'open source.\\
	Je confirme donc mon hypothèse sur le fait que l'on entend pas parler de l'open source pour le grand public car c'est le meme ressenti qui est partagé par les personnes interviewé et ayant répondu au questionnaire.\\

	Pour autant lors de mes différents échanges, je m'aperçoit que la cible du grand public est erronée. En effet, l'open source est généralement à destination des développeurs, des entreprises et dans ce monde là, on est normalement sensibilisé à l'open source.\\

	C'est une erreure d'interprétation que j'ai eu en faisant l'amalgame du consommateur et de l'utilisateur final du produit. Les utilisateurs finaux (ou grand public) ne sont pas forcément développeurs, et les produits open source ne sont pas tous des logiciels à destination de ces utilisateurs.\\

	L'open source s'adresse donc principalement à son consommateur qui est généralement développeur et qui, pour des besoins d'entreprise ou personnel, va utiliser ces briques logicielles afin d'éviter de réinventer la roue.\\

	Ainsi s'il n'est pas nécessaire de sensibiliser le grand public, il n'en est pas de même pour les entreprises.

	\subsection{Sensibiliser l'entreprise et le contributeur}

	En entreprise, le sujet de l'open source est considéré comme très important, on en entend parler, on l'utilise même beaucoup et c'est ce que j'ai pu constater également de mon coté mais aussi par les échanges que j'ai eu auprès des interviews.\\

	Les personnes concernés ont acquiescé le fait que l'entreprise à besoin de l'open source, qu'elle s'attend à ce que les développeur s'y connaissent sur le sujet... Pour autant, ils ne souhaitent pas forcément apporter leur pierre à l'édifice.\\

	Aujourd'hui, il est donc important de sensibiliser les entreprises et de trouver le moyen de les faire contribuer au monde du logiciel ouvert.

	En effet, la contribution de l'entreprise est un axe exponentiel de croissance pour l'open source.\\

	Une entreprise qui contribue en ouvrant son code, sera sensible au autres entreprises qui en ont fait de même. Il sera alors plus facile pour elles de permettre à leurs développeur de contribuer à l'open source tant pour améliorer leur propres code en interne que pour utiliser celui des autres et l'adapter à ses besoins.\\

	Le développeur contributeur sera donc sensibilisé à son tour.

\paragraph{Réponse à l'hypothèse}


