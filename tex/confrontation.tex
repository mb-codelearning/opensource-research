\chapter{Confrontation} 

\section{Promotion de l'open source}

	Ma première hypothèse concerne le manque de promotion, de mise en avant des produits open source, et du manque d'outil disponible sur les plateforme d'hébergement de code open source pour en faire le produit phare de tout les développeurs et entreprises du logiciel.

	\subsection{Des plateformes améliorables tout de même}

		Dans mon étude autour de l'open source, je me suis aperçu de la multitude des plateformes disponibles pour présenter les projets open sources et permettre la contribution. Malgré celà, il me semblait nécessaire de mettre en place une interface pour accueillir les potentiels futur contributeur, une vitrine permettant de visualiser le produit.\\

		Avec l'étude terrain et les résultat à mes questions autour de ce sujet, je me suis aperçu que les plateformes n'ont pas nécessité à promouvoir le projet car l'aspiration à la contribution n'est pas lié à la motivation et au marketing qui gravite autour. C'est sur besoin qu'a le consommateur à utiliser une spécificité du logiciel open source, et donc à communiquer son idée, ses besoins pour  et orienter le développement du produit vers son besoin.\\

		Ainsi mon hypothèse sur l'interface de promotion des projets open source n'est pas valide.
		Néanmoins il s'en est dégagé un besoin crucial d'améliorer la communication entre toutes les parties prenantes du projet open source.

	\subsection{Le marketing de l'open source}

		Au cours de mon étude, j'ai découvert la promotion inhérente à l'open source dans les livres blancs mis à disposition par SMILE, j'ai eu tendance à penser que le marketing 3.0 qui se fait intrinsèquement à la présence du projet sur les plateformes d'hébergement, par l'ouverture du code et donc le partage de celui-ci suffit.\\

		Dans l'étude terrain et notamment lors de mon interview avec un éditeur open source, je me suis rendu compte qu'il est tout de même nécessaire de réaliser un marketing classique autour du produit à promouvoir afin de se faire une place sur le marché.

\paragraph{Réponse à l'hypothèse\\}

	Mon hypothèse concernant la nécessité de mettre en place une promotion à l'aide de vitrine sur les plateforme n'est pas valide, toutefois, un marketing pour promouvoir son produit peut etre réalisé en dehors pour attirer les entreprises et consommateurs.

\section{Optimisation des ressources}

	\subsection{Business model de l'open source}

		La meilleure gestion de projet open source que j'ai pu déniché s'apparentait au modèle noyau / extension qui prévalait les conflits entre l'éditeur et les besoins communautaire. Egalement j'ai pu découvrir le management participatif au travers d'ouvrages sur le sujet que je souhaitais mettre en avant dans la gestion de projet open source.

		En confrontant mon hypothèse à la vision des personnes interrogés et interviewé je m'aperçoit que l'aspect hiérarchique horizontale en ressort bel et bien ce qui confirme celle-ci sur la gestion idéale des projet open source. \\

		Néanmoins, je garde à l'esprit qu'il n'y a pas vraiment de business model applicable car celui-ci est en constante définition tout au long du cycle de vie du projet.

	\subsection{Gestion des ressource humaines}

		Dans les différents ouvrage et mes recherches effectué, je souhaitais profiter de l'aspect communautaire et donc avoir de nombreuses ressource humaines, pour organiser celà à travers un cerveau collectif et donc ajouter un semblant de gestion d'équipe et de projet.\\

		Ceci étant avec les différents entretiens et échanges je m'aperçois que ma réflexion semble erronnée.
		En effet j'en ai conclus que l'open source se doit de se faire par le bon vouloir des personnes qui est souvent guidé par leur besoin de développer un module en particulier.\\

		Ainsi l'éditeur ne peut avoir aucune mainmise sur les contributeurs et la gestion de leur participation au projet.
		Il apparait néanmoins que la mise en place de guide de contribution, faciliter les premières contributions et améliorer la communication de l'éditeur et de la communauté est acceptable pour répondre à ce besoin d'optimisation. 

\paragraph{Réponse à l'hypothèse\\}
	
	Je valide donc à 70\% mon hypothèse sur l'optimisation des ressources disponible car il est possible d'améliorer le rapport humain afin de motiver et faciliter la contribution, en gardant un bémol sur l'impossibilité de gérer et d'attribuer les contributeurs à des tâches.

\section{Envies et besoins de contribuer}

	Ma troisième hypothèse traite du manque de sensibilisation qui plane sur l'open source, faisant de celui-ci un sujet de malentendu, d'incompréhension ou d'ignorance par les consommateurs potentiel. Si l'on sensibilisait plus le consommateur de l'open source sur l'importance de leur contribution, alors le blason de l'open source en serait redoré et attirerait encore plus de monde.

	\subsection{Sensibiliser le grand public}

		Lors de mes recherches autour de l'open source et en comparant avec mon vécu dans mes entreprises et à l'école, je me suis aperçu du manque de sensibilisation sur le vaste sujet qu'est l'open source.\\

		Je confirme mon hypothèse sur le fait que l'on entend pas parler de l'open source pour le grand public car c'est le meme ressenti qui est partagé par les personnes interviewé et ayant répondu au questionnaire.\\

		Pour autant lors de mes différents échanges, je m'aperçoit que la cible du grand public est erronée. En effet, l'open source est généralement à destination des développeurs, des entreprises et dans ce monde là, on est normalement sensibilisé à l'open source.\\

		C'est une erreure d'interprétation que j'ai eu en faisant l'amalgame du consommateur et de l'utilisateur final du produit. Les utilisateurs finaux (ou grand public) ne sont pas forcément développeurs, et les produits open source ne sont pas tous des logiciels à destination de ces utilisateurs.\\

		Ainsi s'il n'est pas nécessaire de sensibiliser le grand public, il n'en est pas de même pour les entreprises.

	\subsection{Sensibiliser l'entreprise et le contributeur}

		En entreprise, le sujet de l'open source est considéré comme très important, on en entend parler, on l'utilise même beaucoup et c'est ce que j'ai pu constater également de mon coté mais aussi par les échanges que j'ai eu auprès des interviews.\\

		Les personnes concernés ont acquiescé le fait que l'entreprise à besoin de l'open source, qu'elle s'attend à ce que les développeur s'y connaissent sur le sujet... Pour autant, ils ne souhaitent pas forcément apporter leur pierre à l'édifice.\\

		Aujourd'hui, il est donc important de sensibiliser les entreprises et de trouver le moyen de les faire contribuer au monde du logiciel ouvert.\\

		En effet, la contribution de l'entreprise est un axe exponentiel de croissance pour l'open source.\\

		Une entreprise qui contribue en ouvrant son code, sera sensible au autres entreprises qui en ont fait de même. Il sera alors plus facile pour elles de permettre à leurs développeur de contribuer à l'open source tant pour améliorer leur propres code en interne que pour utiliser celui des autres et l'adapter à ses besoins.\\

		Le développeur contributeur sera donc sensibilisé à son tour.

	\subsection{L'expression du besoin}

		Dans mes recherches, il ressortait un manque d'expression du consommateur et de compréhension de leur besoin de la part de l'éditeur.\\

		Après l'étude terrain, je me suis aperçu que les consommateur ne se plaignent pas de l'écoute de l'éditeur concernant leur besoin mais néanmoins il apparait certain blocage à l'open source comme le manque de documentation.\\

		Ainsi les besoins sont exprimé mais la communication sur les produit open source laisse perplexe.

\paragraph{Réponse à l'hypothèse\\}

		Sensibiliser à l'open source les entreprises et offrir l'opportunité au consommateur de communiquer autour de ses besoins est donc une hypothèse valide à 90\% pour améliorer la vision de l'open source et le valoriser, néanmoins l'utilisateur final, non développeur n'a pas véléité à être sensibilisé au sujet.










