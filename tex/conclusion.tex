\chapter{Conclusion}

\paragraph{La thèse professionnelle\\}

	L'open source est un vaste océan qui habite le monde de l'informatique. Très utilisé et pourtant peu enseigné, il est intéressant de s'en approcher pour en comprendre les fondements essentiels. 

	Cette thèse professionnelle m'a permis non seulement d'en avoir une vision claire et précise mais également d'en comprendre les limites et les possibilités qui gravitent autour.

	L'intérêt que l'on peut avoir a y contribuer ou tout simplement mieux choisir ses futures briques logicielles qui feront parti intégrante de notre travail.

	Ne pas réinventer la roue, offrir à chacun la possibilité de répondre spécifiquement à son besoin et le faire partager, ce sont les valeur inhérente à l'open source

	J'ai souhaité à travers ma problématique, y apporter un peu de moi à travers mes réflexions personnelles sur l'utilisation de l'intelligence collective et les alternatives au management habituel.Même si cette gestion fine d'un projet open source n'est pas envisageable compte tenu de la liberté qui en découle, il apparait tout de même que l'humain et ses besoins technologiques est au coeur de la raison de l'open source et qu'il est donc primordial de mettre la communication en première ligne dans un projet open source.

\paragraph{Mon avenir\\}

	Dans un futur plus ou moins proche, je désire parvenir au poste d'architecte logiciel.\\
	En effet, lors de mes précédentes années de travail, j'ai pu rencontrer et travailler aux cotés de personnes sensibles à l'architecture logicielle qui m'ont transmise le goût de la réflexion et de l'investissement inhérents à ce métier.

	Pour aspirer à ce futur, la route est longue et j'ai donc réalisé ma roadmap vers mon métier cible.\\
	Premièrement récupérer les briques dont je n'ai pas eu la chance de découvrir dans mes années de travail et d'écoles comme l'algorithmie.
	Ensuite développer mes connaissances autour du développement logiciel, architecture serveur, gestion de projet, sécurité.
	Egalement, monter en compétences sur les différents blocs de communication nécessaire pour le métier d'architecte logiciel.

	J'attend donc avec impatience la fin (toute proche) de ces années d'études pour me plonger dans le coeur de cette aventure vers l'architecture logiciel, aventure dont vous entendrez peut être conter le récit.