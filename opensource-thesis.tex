
\documentclass[a4paper, 11pt]{report}
\usepackage{graphicx}
\usepackage{mystyle}
\usepackage{hyperref}

\newcommand\blankpage{%
    \null
    \thispagestyle{empty}%
    \addtocounter{page}{-1}%
    \newpage}

\geometry{% margin settings
	paper=a4paper, 
	inner=2.0cm, % Inner margin
	outer=2.0cm, % Outer margin
	bindingoffset=0.0cm, % Binding offset
	top=2.0cm, % Top margin
	bottom=2.0cm, % Bottom margin
	includeheadfoot,
	headsep=5mm,
	footskip=1.8cm
}

\setlength{\headheight}{14pt}
\fancypagestyle{plain}{
\fancyhead[L]{Valorisation de l'open source}
\fancyhead[R]{Matthieu Balondrade - MSI 2019}
\fancyfoot[L]{\includegraphics[width=3cm]{./img/cesi.png}}
\fancyfoot[R]{\includegraphics[width=3cm]{./img/docdoku.png}}}
\pagestyle{plain}
\makenoidxglossaries


\definecolor{witchhaze}{RGB}{255, 246, 143}
\definecolor{burntorange}{RGB}{211, 84, 0}
\definecolor{silver}{RGB}{131, 135, 138}


\newglossaryentry{oeuvres derivees}
{
        name=œuvres dérivées,
        description={Est considéré comme œuvres dérivées, tout programme A qui en modifiant quelques lignes de code donne un programme B ou un programme A qui appelle les fonctions d'un programme B de sorte que l'un ne puisse pas être pleinement utile sans l'autre.}
}
\newglossaryentry{latex}
{
        name=latex,
        description={Is a mark up language specially suited for 
scientific documents}
}
\newglossaryentry{open source}
{
		name=open source,
		description={La désignation open source, ou « code source ouvert », s'applique aux logiciels (et s'étend maintenant aux œuvres de l'esprit) dont la licence respecte des critères précisément établis par l'\acrfull{osi}, c'est-à-dire les possibilités de libre redistribution, d'accès au code source et de création de travaux dérivés.
	}
}
\newglossaryentry{Creative Commons}
{
	name=creative commons,
	description={
		https://creativecommons.org/licenses/by-nd/2.0
	}
}
\newglossaryentry{framework}
{
	name=framework,
	description={
		Un framework est un ensemble d'outils et de composants logiciels organisés conformément à un plan d'architecture et des patterns, l'ensemble formant ou promouvant un "squelette" de programme, un canevas. Il est souvent fourni sous la forme d'une bibliothèque logicielle et accompagné du plan de l'architecture cible du framework.
		Un framework est conçu en vue d'aider les programmeurs dans leur travail. L'organisation du framework vise la productivité maximale du programmeur qui va l'utiliser.
	}
}
\newglossaryentry{apache}
{
	name=apache,
	description={
		L'Apache Software Foundation (ASF) est une organisation à but non-lucratif qui développe des logiciels open source sous la licence Apache, dont le renommé serveur web Apache HTTP Server. Elle a été créée en juin 1999 dans le Delaware aux États-Unis.
	}
}
\newglossaryentry{scalable}
{
	name=scalable,
	description={
		La faculté d'un logiciel à évoluer dans le temps de façon pérenne en augmentant ses revenus sans avoir à augmenter les effectifs, et donc les coûts.
	}
}
\newglossaryentry{web 3.0}
{
	name=web 3.0,
	description={
		Le web 3.0 est considéré comme l'évolution du Web 2.0 qui correspondait à l'ère des réseaux sociaux. Même si la définition n'est pas fixée et encore débattue, on peut dire du web 3.0 qu'il appartient à l'ère de l' « internet des objets », ou \acrfull{iot}. Il regroupe toutes les nouvelles technologies apparues et déployées largement aujourd'hui : cloud computing, intelligence artificielle, deep learning.
	}
}
\newglossaryentry{conteneurisation}
{
	name=conteneurisation,
	description={
		À l'image de la conteneurisation qui désigne le terme de transporter la marchandise par des conteneurs, dans le domaine du logiciel, il s'agit d'un procédé permettant d'exécuter des applications dans une zone isolée appelé conteneur. L'application stocke les données dans le conteneur et celui-ci se connecte au noyau d'un système d'exploitation. Il n'est pas nécessaire d'installer un nouveau système d'exploitation, car un seul est défini comme la base qui possède l'ensemble des conteneurs.
	}
}

\newglossaryentry{fork}
{
	name=fork,
	description={
	Un fork (terme anglais signifiant « fourche », « bifurcation », « embranchement ») est un nouveau logiciel créé à partir du code source d'un logiciel existant lorsque les droits accordés par les auteurs le permettent : ils doivent autoriser l'utilisation, la modification et la redistribution du code source.
	}
}

\newglossaryentry{mainliner}
{
	name=mainliner,
	description={
	Centraliser ses contributions en les ramenant dans le projet originel.
	}
}
\newglossaryentry{issues}
{
	name=issues,
	description={
	De l'anglais, actions (correctif, bug, tâches) à réaliser dans le code logiciel  
	}
}
\newglossaryentry{leechers}
{
	name=leechers,
	description={
	En informatique, un leecher ou leech (de l'anglais leech, sangsue) est un utilisateur qui profite d'un système sans rien apporter en retour.
	}
}


\newacronym{osi}{OSI}{Open Source Initiative}
\newacronym{floss}{FLOSS}{Free/Libre and Open Source Software}
\newacronym{fsf}{FSF}{Free Software Foundation}
\newacronym{bsd}{BSD}{Berkeley Software Distribution}
\newacronym{gnu gpl}{GNU GPL}{GNU General Public Licence}
\newacronym{saas}{SaaS}{Software as a Service}
\newacronym{ca}{CA}{chiffre d'affaire}
\newacronym{ssll}{SSLL}{Sociétés de Service en Logiciel Libre}
\newacronym{ssii}{SSII}{Sociétés de Service en Ingénierie Informatique}
\newacronym{iot}{IOT}{Internet of Things}
\newacronym{cnll}{CNLL}{Conseil National du Logiciel Libre}



\title{\Huge \color{burntorange}{Valorisation de l'open source en tant qu'éditeur}}
\author{\Large Matthieu \bsc{Balondrade} \\MSI 2019 }
\date{\Large 5 Février 2020}



\begin{document}
\begin{titlepage}
	\centering
	\includegraphics[width=0.3\textwidth]{./img/cesi.png}\par\vspace{1cm}
	{\scshape\Large MSI 2019\par}
	\vspace{3cm}
	{\huge\bfseries \color{burntorange}{Valorisation de l'open source en tant qu'éditeur}\par}
	\vspace{2cm}
	{\Large\itshape Matthieu Balondrade\par}\vspace{2cm}
	{\large 5 Février 2020\par}
	\vfill

% Bottom of the page
	\includegraphics[width=0.3\textwidth]{./img/docdoku.png}\par\vspace{2cm}\par

	Diffusion limitée au CESI, reproduction interdite.\par


\end{titlepage}
\afterpage{\blankpage}
\tableofcontents
\listoffigures
\printnoidxglossaries
\printglossary[type=\acronymtype]


\chapter*{{Preface}}
	\section*{Remerciements}

	Ce travail m'a été rendu possible grâce à de nombreuses personnes. Je souhaite donc les en remercier avant de vous le faire partager.\\

	Mon premier remerciement s'adresse à Florian \bsc{Gasc}, source de motivations et d'inspirations qui m'a donné l'envie de persévérer pour devenir architecte logiciel. Merci d'avoir pu répondre à mon questionnaire.

	Je souhaite remercier les membres de DocDoku, mon entreprise actuelle sans laquelle je n'aurais pas eu l'idée de ce sujet.

	Merci à Quentin \bsc{Cazelle} pour nos échanges sur l'open source et l'interview réalisée.
	Je remercie Rémi \bsc{Buhler} qui à pu m'apporter sa vision sur l'open source.

	Également, je remercie mon tuteur, Florent \bsc{Garin}, qui m'a beaucoup apporté dans ma compréhension de ce sujet et qui m'a apporté une interview riche en connaissances.

	Merci à l'équipe Framasoft et leur communauté, merci pour les échanges, l'apport de connaissances sur l'open source et notamment les différences avec le libre et merci pour votre outil de formulaire !

	Un grand merci à Olivier \bsc{Mignial} pour notre interview et sa société, Smile, qui est riche en contenu open source et m'a aidé à mieux comprendre ce sujet.

	Je remercie fortement toutes les personnes qui ont contribué à mon questionnaire et ont pris le temps d'y répondre convenablement.\\

	Enfin, je vous remercie vous, lecteurs, qui m'avez donné l'occasion de m'épanouir dans l'écriture de cette thèse professionnelle et de vous faire partager les richesses de mes réflexions.

	\includepdf[pages=-]{confidentialitecesi.pdf} % Fiche de confidentialité
\chapter*{Abstract}

\section*{English}
	
	\paragraph*{Introduction\\}

	This document aims to expose the different arguments on how to enhance the open source work as an editor and promote open source to be the main solution picked by potential consummers.

	Open source is very important in IT and moreover is a school of thought. Understanding how and why it matters will increase the number of contributions and help the IT's world. That is why I offer you a first step into open source from the editor's side.

	My hypotheses are that we can improve the promotion made to the open source from the hosting platform. We can optimize the human ressources that contribute to the project by attributing their work and specilizing them and also that we need to raise awareness of the world about the open source importance.

	\paragraph*{First part\\}

	I did my research, over all, about the open source and specificaly the aspect that can be improved in order to make it the best solution for its consummers.
	My studies gather informations about the open source's consummer: it expectations, it needs and requirements. You'll also find more about the editor and the business model that works with the open source project.

	In order to improve the open source community sharing work, I did research about the collective intelligence from books and the horizontal management of project.

	Lastly, I found some aspect on what kind of marketing could be done to promote the open source solutions.

	Once this research done, I started to ask people's opinion about my thought related to open source.

	\paragraph*{Second part\\}

	I did some interview with people concerned by open source. In addition a form was sent and treated by 40 persons.

	The answers reflect multiple improvements that can be done on the communication aspect in general, whether it is for the consummer's expectation that needs to be heard or the editor that should guide contributions.

	It has also enlightened my vision about the collective aspect and project management of the open source work. I understood that all contributions made to the open source are on a voluntary basis, hence the discrepency between my expectations to attribute specific work to contributors on the project and their free will.

	Make companies and developers aware of the importance of open source and its interests is a fact that emerged from this study.

	Finally, despite the marketing aspect coming from the "open" part of the open source, a classic marketing to promote the product is still required.

	I was now able to compare my opinion to people's one in order to answer my hypotheses and give an anwser for editors to enhance the open source.

	\paragraph*{Hypotheses answered\\}

	This confrontation revealed that my hypothesis about the awareness of people that need to be raised is correct but does concern more the companies and the developers rather than the entire world.

	I was wrong about the optimization of the human ressources by attributing their work using collective intelligence and managing them because contributors can not be managed.

	Hosting platforms can be improved, not in the way that I expected, but on the documentation aspect that will help people to contribute easily. So my hypothesis about that is partially valid

	\paragraph*{Conclusion\\}

	To begin your journey in open source as an editor, this study will show you some levers that should be considered such as communications, awareness of people and companies and guides for first contributors.

\section*{Français}

	\paragraph*{Introduction\\}

	Ce document a pour but d'exposer les différents arguments sur la façon d'améliorer l'open source en tant qu'éditeur et de promouvoir celui-ci afin d'en faire la solution principale pour les consommateurs.

	Je vous parle de l'open source car il s'agit d'un sujet très important pour l'informatique mais également d'un mouvement de pensée.
	En comprenant le pourquoi et le comment de son importance, cela augmentera peut-être le nombre total de contributions et améliorera le monde de l'informatique. C'est pourquoi je vous parle de l'open source.

	Mon hypothèse est que nous pouvons améliorer la promotion faite à l'open source à partir de la plateforme d'hébergement. Nous pouvons optimiser les ressources humaines qui contribuent au projet en attribuant leur travail et en les spécialisant. Egalement, je pense que nous avons besoin de sensibiliser le monde à l'importance de l'open source.

	\paragraph*{Première partie\\}

	J'ai fais mes recherches sur l'open source et plus particulièrement sur ce qui peut être amélioré pour en faire la meilleure solution aux yeux des consommateurs.

	Mes études rassemblent des informations sur le consommateur de l'open source : ses attentes, ses besoins et ses exigences. Vous y trouverez des informations sur l'éditeur et le modèle économique qui fonctionne avec le projet open source.

	Afin d'améliorer le travail de partage de la communauté open source, j'ai fait des recherches sur l'intelligence collective issue d'ouvrages et la gestion horizontale de projet.

	Enfin, j'ai trouvé un aspect sur le type de marketing qui pourrait être fait pour promouvoir les solutions open sources.

	Une fois cette recherche effectuée, j'ai commencé à demander aux gens leur opinion sur le sujet.

	\paragraph*{Deuxième partie\\}

	J'ai fait quelques interviews avec des personnes concernées par l'open source. De plus, un formulaire a été envoyé et traité par 40 personnes.

	Les réponses reflètent les multiples améliorations qui peuvent être apportées sur l'aspect communication en général, que ce soit pour les attentes du consommateur qui doivent être entendues ou de l'éditeur qui doit guider les contributions.

	Cela a également éclairé ma vision sur l'aspect collectif et la gestion de projet du travail open source. J'ai compris que toutes les contributions faites à l'open source le sont sur une base volontaire, d'où le décalage entre mes attentes d'attribuer un travail spécifique aux contributeurs sur le projet et leur libre-arbitre.

	Faire prendre conscience aux entreprises et aux développeurs de l'importance de l'open source et de ses intérêts est un fait qui ressort de cette étude.

	Enfin, malgré l'aspect marketing venant de la partie "open" de l'open source, un marketing classique pour promouvoir le produit open source est encore nécessaire.

	J'ai donc pu comparer mon opinion à celle des gens afin de répondre à mes hypothèses et de donner une réponse aux éditeurs pour améliorer l'open source.

	\paragraph*{Réponses aux hypothèses\\}

	Cette confrontation a révélé que mon hypothèse sur la prise de conscience des personnes à sensibiliser est correcte, mais qu'elle concerne plus les entreprises et les développeurs que le monde entier.

	Je me suis trompé sur l'optimisation des ressources humaines en pensant attribuer leur travail par l'intelligence collective, car les contributeurs ne peuvent pas être managés.

	Les plateformes d'hébergement peuvent être améliorées, pas de la manière que j'attendais, mais plus sur l'aspect documentation qui aidera les gens à contribuer facilement. Mon hypothèse à ce sujet est donc partiellement valable.

	\paragraph*{Conclusion\\}

	Pour débuter votre parcours dans l'open source en tant qu'éditeur, cette étude vous montrera quelques leviers à considérer tels que la communication, la sensibilisation des personnes et des entreprises et des guides pour les premiers contributeurs.






\chapter{Introduction}

	\section{Contexte}

		\paragraph{Développement logiciel \\}
		
			 Le contexte de cette thèse se focalise sur le monde et l'environnement de développement logiciel au cœur de l'informatique. L'\Gls{open source} aujourd'hui, peut s'étendre aux œuvres de l'esprit qui ne seront pas traités particulièrement, mais désignent les logiciels dit "ouverts".
		
		\subsection{Introduction à l'open source}

			Afin de lever l'ambiguïté entre les logiciels "libres" (en anglais "free", pouvant désigner un logiciel gratuit, ou libre) et les logiciels "ouverts", l'expression "\gls{open source}" est apparue en 1998 par Christine \bsc{Peterson} du Foresight Institute.
			Richard \bsc{Matthew Stallman} défend le terme de \textit{free software} à travers son organisme la \acrfull{fsf}
			Eric \bsc{Raymond} et Bruce \bsc{Perens} créent en 1998, l'\acrfull{osi}, qui délivre le label "\acrshort{osi} approved" aux licences qui satisfont aux critères définis dans l'Open Source Definition.

			\begin{center}
				\textit{
				\textquote{
					L'Open Source permet une méthode de développement de logiciels qui exploite la puissance de l'évaluation par les pairs distribuée et la transparence des processus. La promesse de l'open source est une meilleure qualité, une meilleure fiabilité, une plus grande flexibilité, des coûts moins élevés et la fin de l'immobilisme prédateur des fournisseurs.} - \acrshort{osi}
				}
			\end{center}

			Un logiciel \textit{open source}, est donc un programme dont le code source est distribué et peut être utilisé, copié, étudié, modifié et redistribué sans restriction.

			Les deux appellations «open source» et «logiciel libre» sont presque équivalentes, mais correspondent à des écoles de pensées différentes. Libre ne signifie pas gratuit.
			Rien n'interdit de faire payer le logiciel bien que l'utilisateur en fonction de la licence pourra le redistribuer librement, on considère donc qu'un logiciel open source est généralement gratuit.
				
		\paragraph{Pourquoi je vous parle de l'open source ?\\}

			L'open source est un outil capital dans le monde de l'informatique.
			En tant que développeur j'ai découvert l'open source au sein de mon entreprise actuelle, DocDoku.
			Celle-ci développe un logiciel open source aidant de nombreuses entreprises.

			Bien qu'important, l'open source n'est pourtant que vaguement évoqué et méconnu, ainsi à tort deux affirmations sont faites:
			
			\subparagraph{« L'open source c'est moins bien »\\}

				L'open source est souvent associé à de mauvais termes. On peut entendre logiciel gratuit, non supporté par une entreprise, développé par le particulier ou des communautés en tant que passe-temps. Un manque de rigueur peut donc s'en dégager et dévaloriser l'image de celui-ci auprès des consommateurs.

			\subparagraph{« L'open source n'a pas besoin de moi »\\}

				L'open source ne se résume pas seulement au développement de code informatique et est un vaste domaine ou chaque individu, quelque soit ses compétences, peut y participer afin de rendre le monde de l'informatique meilleur.\\

			Malgré le projet que développe notre entreprise sous open source, je n'ai pas réussi à obtenir des réponses concrètes sur ce qu'est l'open source, ce que cela implique, auprès de mes collègues de travail et de formation. A ceci, l'on peut ajouter que pour beaucoup, l'open source est signe de stabilité, fiabilité, sécurité.\\

			En plus du manque de connaissance sur le sujet par moi-même et mon entourage, je constate ainsi qu'il existe encore beaucoup d'amalgames autour de l'open source, des défauts et abus de langage.\\
			Les institutions et entreprises que j'ai pu fréquenter dans le monde de l'informatique ne sensibilisent pas assez les développeurs actuels et en devenir sur l'open source. Les avantages qu'offrent celui-ci,dont l'aspect juridique, les conditions d'utilisation et bien d'autres caractéristiques propres à l'open source ne sont pas enseignées.\\

			Je pars donc du principe que l'open source ne fait pas assez de "bruit" dans le monde, ainsi je souhaite développer à travers cette thèse, la curiosité, la passion et l'envie de nous tourner vers l'open source, d'y contribuer et d'en faire la solution privilégiée des développeurs et des consommateurs occasionnels en me positionnant du coté des éditeurs.

		\subsection{De l'importance de l'open source}

				Une étude auprès de 950 entreprises dans l'informatique a été réalisé en 2019 par la fondation Redhat qui a permis de statuer sur l'évolution de l'open source. La moitié de ces entreprises sont situées aux États-Unis, le reste est réparti entre l'Asie Pacifique, le Royaume-Uni et l'Amérique Latine.

				\paragraph{L'open source à travers le monde\\}

					L'open source est utilisé par beaucoup de personnes et ce à travers le monde selon une étude (voir Annexe 1) sur la diffusion de l'open source à travers les frontières.
					Son utilisation est versatile, les entreprises interrogées l'utilisent pour différents secteurs 

					\begin{itemize}[label=\textbullet, font=\LARGE \color{burntorange}]
					\item Modernisation des infrastructures
					\item Transformation digitale
					\item Développement d'applications
					\item DevOps
					\item Intégration dans les applications
					\item Modernisation des applications
					\end{itemize}

				\paragraph{Les domaines informatiques de l'open source dans l'entreprise\\}

					Les nouveaux domaines du numérique en vogue ainsi que certains reservés auparavants aux logiciels propriétaires s'ouvrent sur l'open source. Ceci inclus notemment les outils de gestion du cloud, security, la sécurité, les données analysées, et le stockage (Voir annexe 2)

				\clearpage
				\paragraph{Les réticences encores présentes sur l'open source\\}

					Malgré l'impact bénéfique de l'open source pour les entreprises, il reste encore quelques inquiétudes planant au dessus de l'open source et des réticence à son utilisation généralisée.

					\begin{figure}[!htb]
						\center
						\includegraphics[scale=0.70]{./img/Barreer_os.png}
						\caption{Causes principales des réticences à l'open source}
						\caption*{\color{silver}Source: redhat.com}					
					\end{figure}										

					J'en conclu qu'à travers le hack et le piratage de produits open sources, les entreprise positionnent la sécurité de l'open source comme une inquiétude majeure.

				\paragraph{L'évolution de l'utilisation de l'open source\\}

					Beaucoup d'entreprises continuent tout de même à utiliser des logiciels propriétaires, mais cette tendance tend à diminuer sur les deux prochaines années. Ceci grâce aux nouvelles technologies provenant du \gls{web 3.0}, notemment la \gls{conteneurisation}, qui est considérée comme un brassage de produits collaboratifs open sources. De nombreuses entreprises aujourd'hui se tournent vers des solutions de conteneurisation, due à l'open source.
					
					\begin{figure}[!htb]
						\center
						\includegraphics[scale=0.60]{./img/osinentreprise.png}
						\caption{Évolution de l'open source dans les années à suivre}
						\caption*{\color{silver}Source: redhat.com}					
					\end{figure}

					\newpage

	\section{Problématique}
		\paragraph*{}

			Je vous propose de traiter la problématique suivante :
			\begin{center}
				\begin{displayquote}
					\textbf{Comment valoriser l'open source, en tant qu'éditeur, et en faire la solution privilégiée des consommateurs ?}
				\end{displayquote}
			\end{center}

		\paragraph*{}
			
			Afin de répondre à cette problématique, je vais m'appuyer sur des hypothèses fondées sur mon vécu en entreprise mais également issues de ma réflexion personnelle dans mon métier de développeur logiciel.

	\section{Hypothèses}

		\paragraph{Promouvoir l'open source\\}

			Selon moi si les plateformes hébergeant du contenu open source et offrant la possibilité d'y contribuer mettaient plus en avant les projets open sources, par une démarche plus marketing et publicitaire à travers l'utilisation d'images et la présentation du projet par vidéo, l'open source serait plus prisé par les consommateurs et l'on constaterait plus de contributions à ces projets.

		\paragraph{L'optimisation\\}

			Si nous améliorons l'organisation de la communauté, en impliquant un peu mieux les personnes sans pour autant compliquer leur participation , nous valoriserons l'open source en le rendant plus performant.

		\paragraph{L'envie de contribuer\\}

			Si toutes les personnes liés à l'informatique était sensibilisées sur l'importance de l'open source et si l'éditeur leur donnait envie de contribuer en leur facilitant la tâche par des guides ou des moyens de s'exprimer plus facilement, l'open source serait la solution idéale à leurs yeux.

	\section{Méthodologie de travail / périmètre}
		\paragraph{Etude de l'open source}

			\subparagraph{Présentation de l'open source\\}

				Afin d'avoir un retour sur mes hypothèses, il m'est nécessaire d'avoir une meilleure vision d'ensemble de l'open source et un périmètre défini.

			\subparagraph{Les éditeurs open source\\}

				Je me positionne du côté des éditeurs, pour les diverses raisons évoquées précédemment, je souhaite donc en savoir plus sur l'éditeur, ses rôles, son périmètre d'action.

			\subparagraph{L'optimisation des ressources\\}
		
				Après avoir fait la lecture et l'analyse personnelle de plusieurs ouvrages dont 2 principaux sur lesquels je m'appuierai : 
		
				\begin{displayquote}
					\textquote{Réfléchissez et devenez riche} - Napoleon \bsc{Hill} - 1937
				\end{displayquote}
				\begin{displayquote}
					\textquote{Oser la confiance} - Bertrand \bsc{Martin} (†), Vincent \bsc{Lenhardt}, Bruno \bsc{Jarrosson} - 1996
				\end{displayquote}
		
				Je souhaite effectuer des recherches sur les moyens d'optimiser les ressources qui définissent l'open source.

			\subparagraph{Le consommateur\\}
				En étudiant le consommateur, nous verrons les besoins et activités relatives à l'open source.

			\subparagraph{Le marché de l'open source\\}
				Enfin nous verrons l'aspect économique associé à l'open source en étudiant son marché.

		\paragraph{Sur le terrain \\}
			Lors de mon étude terrain je consoliderai mes recherches dans les domaines cités précédemment pour pouvoir les confronter.

		\paragraph{L'analyse et les résultats \\}
			Je transposerai alors mes études et relevés afin de valider ou réfuter mes hypothèses et apporter une réponse acceptable à ma problématique.
\chapter{\color{burntorange}{Etat de l'art}} %20 pages

	% --------------------------------------------------%
	%     												%
	%             Présentation de l'open source			%
	%       										    %
	% --------------------------------------------------%

	\section{Présentation de l'open source} % in progress

		\subsection{L'open source aujourd'hui}

			\subsubsection{Faire de l'open source c'est ...}

				\paragraph{Un mouvement de pensée\\}

					Le logiciel open source est une affaire de liberté. N'importe quel programme devrait être utilisable, modifiable et redistribuable. Les logiciels non libres, "propriétaire", porte atteinte à cette philosophie. Le logiciel libre n'est pas une alternative ou bien un autre business model mais une lutte pour la liberté.

					Un grand mouvement est initié par Richard \bsc{Stallman}, le père fondateur de la \acrshort{fsf} qui soulève un patrimoine extraordinaire de logiciel mis à disposition de tous. Il s'agit d'une véritable révolution pour ce mouvement de pensée profond ou la solidarité sociale et l'entraide en sont les pilliers.

				\paragraph{Un modèle de développement\\}

					L'utilisation d'un modèle de développement open source communautaire est pour Eric \bsc{Raymond} un moyen de démontrer la \textit{supériorité} des logiciels réalisés. Plus que des valeurs éthiques, c'est par la création de ce mouvement avec la fondation de l'\acrshort{osi} qu'Eric \bsc{Raymond} espère imposer l'open source. Pour certain, ceci apparaît comme une \textit{opération marketing}, pour d'autre comme Richard \bsc{Stallman}, il n'est pas permis de jeter les valeurs fondatrices notemment la \textit{liberté}. 

				\paragraph{Une dimension humaniste et un patrimoine\\}

					L'open source permet avant tout d'offrir une chance pour les informaticiens futur de ne pas repartir de zéro, ne pas ré-inventer la roue. Chaque participation de l'Humain apporte sa pierre à l'édifice.
					Seulement 10\% d'un code source est issue de notre création pour 90\% de réutilisation de code issus de système d'exploitation, \gls{framework}, et autres composants.

					C'est ici la valeur ajoutée de l'open source. L'informatique progresse essentiellement car le socle de code qui constitue notre patrimoine s'agrandit.

				\paragraph{Respecter des droits\\}

					Je ne peux pas vous parler d'open source sans mentionner les licences et les différents droits inhérent à cette mouvement.

					Les programmes open source ne sont pas des programmes « sans licences » comme on l'entend parfois. C'est au contraire leur licence qui les fait open source. Ils ne sont pas non plus dans le domaine public, c'est à dire n'appartenant à personne en particulier, ou du moins exempts de droits patrimoniaux.\\

					Lorsqu'un développeur écrit un programme, il en détient les droits d'auteur, le « copyright ». Dans certains cas, ce peut être l'entreprise qui l'emploie qui en détient les droits. Et ce copyright peut être vendu, comme bien immatériel, d'une entreprise à une autre. \\

					Le détenteur du copyright est libre de définir l'utilisation qui peut être faite de son programme : 

					\begin{itemize}[label=\textbullet, font=\LARGE \color{burntorange}]
						\item Il peut le garder pour lui, en interdire l'utilisation à qui que ce soit.
						\item Il peut vendre ses droits à un tiers, personne physique ou morale.
						\item Il peut utiliser son droit d'auteur pour préciser les conditions qu'il pose à l'utilisation de son programme. Il écrit ces conditions dans les termes de la licence d'utilisation.
					\end{itemize}
				
					Il est donc important de bien assimiler la logique suivante : à la base de l'open source il y a la licence, et la licence n'existe qu'à partir du droit d'auteur.\\

					Ainsi tous les logiciels open source ont un propriétaire, ils ne sont pas « à personne », ni même « à tout le monde ». Dans certains cas, ce propriétaire peut être une fondation à but non lucratif, ou bien ce peut être une entreprise commerciale ordinaire. Il peut s'agir aussi de plusieurs coauteurs, en particulier à la suite de contributions successives.\\

					Le détenteur des droits est libre de fixer les conditions de licence, il est libre d'en changer même, et il est libre d'y faire des aménagements ou exceptions, ou de diffuser à certains selon une licence, à d'autres selon une autre licence.\\

					Celui qui reçoit le programme, en revanche, n'est pas libre. Il est lié par les termes de la licence. Certes il n'a pas signé de contrat, mais la licence lui a été bien énoncée, et elle stipule qu'il n'a le droit d'utiliser le programme que sous telles et telles conditions. S'il refuse ces conditions, il n'a pas le droit d'utiliser le programme. 

					Je vais détailler légèrement cette partie car elle est également une explication des différents modes de pensées et communautée qui ont un impact sur l'évolution de l'open source et mes hypothèse.

					Il y a deux grandes familles de licences open source : la famille \acrshort{bsd} et la famille \acrshort{gnu gpl}. le terme de licence copyleft apparait pour les premières. « Copyleft » est bien sûr un jeu de mot en référence au « copyright ». Il ne signifie pas pour autant un abandon de droits.\\

					\subparagraph{Les licences \acrshort{bsd}\\}

						La licence \acrfull{bsd} autorise n'importe quelle utilisation du programme, de son code source et de travaux dérivés. Le code sous licence \acrshort{bsd} peut en particulier être utilisé, intégré à des logiciels sous licence non open source. Microsoft a repris du code sous licence \acrshort{bsd} dans Windows, et que MacOSX est basé sur FreeBSD, une distribution unix \acrshort{bsd}.\\

						La seule contrainte spécifique à cette licence est l'interdiction de chercher à tirer avantage de la dénomination de l'auteur, ici l'Université de Berkeley.\\

						Moins de contrainte, plus de liberté: les programmes sous licence \acrshort{bsd} sont quasiment dans le domaine public.\\

						Dans la famille \acrshort{bsd}, on trouve aussi la licence MIT, et la licence \Gls{apache}. Les différences entre ces différentes licences sont de l'ordre du détail.

					\subparagraph{Les licences GNU/GPL\\}

						La licence \acrfull{gnu gpl} est utilisée par 70\% des programmes open source.\\ 
				
						La licence \acrshort{gnu gpl}, se caractérise principalement par son article 2, qui énonce le droit de modifier le programme et de redistribuer ces modifications, qui constituent des œuvres dérivées, à la condition que ce soit sous la même licence GPL. \\
				
						C'est ce que certains appellent le caractère viral de la licence : elle se communique aux travaux dérivés. Mais je parlerai ici de donnant-donnant. \\
				
						\textbf{Qu'est-ce exactement qu'une œuvre dérivée et qu'entend-on par distribuer ?}\\

						Vous verrez que la définition est vaste et pas toujours évidente à perçevoir.\\

						Est considéré comme une oeuvre dérivée:

						\begin{itemize}[label=\textbullet, font=\LARGE \color{burntorange}]
							\item Si l'on prend un programme A, que l'on modifie des lignes de codes pour obtenir un programme B.
							\item Egalement, utiliser un programme A depuis un programme B, en fesant appel à certaines fonctions (on parlera de \text{\textit{«link»}}) est considéré comme oeuvre dérivée.
							\item Comme il existe beaucoup de façon d'appeler un programme A depuis un programme B, on considère donc que si un programme B ne peut pas fonctionner de manière utile sans A alors il est une oeuvre dérivée.
						\end{itemize} 

						Lorsque l'on « Distribue » un logiciel open source, on livre l'ensemble du code source aux personnes concernées. Commercialiser son programme c'est le distribuer.\\

						A l'inverse il n'est pas considéré comme distribué une oeuvre dérivée qui est utilisé au sein de son entreprise constructrice.Proposer un oeuvre dérivée à travers son service en ligne, comme les \acrfull{saas}, n'est pas considéré comme \textit{distribué}.\\
						Je vous détaille cela plus précisemment dans la licence AGPL\\

						En synthétisant, l'idée de la licence GPL est que, en tant qu'auteur ou propriétaire d'un programme, je vous donne le droit de l'utiliser et d'utiliser ses sources à condition que vous en fassiez autant.\\

						Ceci à pour effet de diviser le monde en deux « camps ».
						Si vous êtes du coté GPL, alors tout le patrimoine open source sous GPL vous est accessible sans restriction.\\

						C'est donc ce que j'appellerai ici du \textbf{donnant-donnant}.

					\subparagraph{La licence AGPL (Affero)\\}
				
						Comme dit précédemment il est possible de prendre un programme sous licence GPL, le modifier et l'utiliser au sein de son organisation sans en livrer les sources. Rien n'empêche non plus d'offrir un service en ligne construit avec une oeuvre dérivée sans diffuser les sources.\\

						A l'ère des services hébérgés de type \acrshort{saas}, ce type d'usage est régulier or il devrait être considéré comme prohibé étant donné qu'il ne respecte pas vraiment le dogme de la FSF.En effet si le programme est accessible via internet, commercialisé généralement mais que les sources ne sont pas disponibles, on devrait parler de distribution.

						Pour répondre à cette controverse, une licence AGPL ou Affero GPL à été créée par la société Affero avec la FSF. Elle ajoute un article qui dit que si le programme initial permettait un accès par le réseau et diffusait ses sources par le réseau, alors le programme dérivé doit en faire de même.\\

						L'article est le suivant:\\

						« Si le programme tel que vous l'avez reçu est prévu pour intéragir avec les utilisateurs au travers d'un réseau, et si, dans la version que vous avez reçue, un  utilisateur intéragissant avec le programme avait la possibilité de demander la transmission du code source intégral du programme, vous ne devez pas retirer cette possibilité pour la version modifiée du programme ou une oeuvre dérivée du programme (...) »

					\subparagraph{Droits d'auteurs\\}

						Qu'en-est-il des droits d'auteurs lors de la réalisation d'une oeuvre dérivée?

						Le droit d'auteur ou copyright sur un programme est une notion quand à elle assez claire contrairement à la notion d'oeuvre dérivée, source d'ambigüité.\\

						Si un programmeur écrit du code que son esprit conçoit, il n'est pas en train de violer un quelconque copyright. Lui-même, ou son employeur, est titulaire des droits d'auteur \textbf{sur son code}.\\
						Plus simplement, on sait quand on enfreint un droit d'auteur.

			\subsubsection{Études et statistiques sur l'open source}

			\paragraph{L'open source à travers le monde\\}

				L'open source est donc utilisé par beaucoup de personnes et ce à travers le monde comme le montre une étude sur la diffusion du libre à travers les frontières.

				%Schéma utilisation de l'open source à travers le monde

			\paragraph{Les domaines de l'open source dans l'entreprise\\}

				Ce relevé indique pour quels domaines métier nous utilisons majoritairement l'open source:

				%Schéma domaine de métier open source

			\paragraph{Les réticences encores présente sur l'open source\\}

				Malgré l'impact bénéfique de l'open source pour les entreprises, il reste encore quelques inquiétudes planant au dessus de l'open source et des réticence à son utilisation généralisée.

				A travers le hack et le piratage de produit open source, les entreprise positionnent la sécurité de l'open source comme une inquiétude majeure.

				% Schéma d'inquiétudes de l'open source pour les entreprises

			\paragraph{L'évolution de l'utilisation de l'open source\\}

				Beaucoup d'entreprises continuent tout de même à utiliser des logiciels propriétaires, mais cette tendance tend à diminuer sur les 2 prochaines années. Ceci grâce aux nouvelles technologies provenant du \gls{web 3.0}, notemment la \gls{conteneurisation}, qui est considéré comme un brassage de produit collaboratif open sources.De nombreuses entreprise aujourd'hui se tourne vers des solutions de conteneurisation, due à l'open source.

				% Schéma d'evolution de l'open source

		\subsection{Les éditeurs open source}

			\subsubsection{Zoom sur l'éditeur}

				L'éditeur, c'est celui qui détient les droits du produit, en assure le développement, la promotion, la diffusion et le support.\\
			
				La seule différence avec l'éditeur de logiciels pour l'éditeur open source est qu'il publie son produit sous licence open source. Sinon l'investissement dans le développement du produit et son marketing est le même qu'un produit propriétaire.
				Ce modèle a été élu pour permettre de briser les position acquise d'oligopoles sur le marché du logiciel.
				Il s'agit donc majoritairement de petites entreprises éditrice qui font du support et du développement du produit leur credos.
				Développer un programme open source coute (un peu) moins cher pour ces entreprises car :

				\begin{enumerate}[font=\color{burntorange}]
		 			\item Elles peuvent s'appuyer sur autant de brique que la licence de son logiciel lui permette.
		 			\item Elles bénéficient de contributions communautaire, que je détaille ultérieurement.
			 		\item Elles possèdent généralement plus de développeur passionné participent à son travail.
				\end{enumerate}

				Nous parlerons plus en détail du modèle économique des éditeurs dans la partie sur le marché de l'open source.
				Généralement, l'éditeur fait le choix de partir sur une licence GPL car elle présente pour eux deux avantages considérables:


				\begin{enumerate}[font=\color{burntorange}]
		 			\item Du fait de sa popularité, elle est parfaitement lisible et compréhensible ce qui la rend gage de tranparence.
		 			\item Elle empêche les autres de se faire de l'argent sur son dos car elle interdit l'intégration du produit dans un développement propriétaire.
				\end{enumerate}

			\subsubsection{L'aspect communautaire}

				L'éditeur open source a à sa disposition une communauté qui pourra l'aider non seulement dans le support sur les ressources open source qu'il utilise mais également au développement et au support de son oeuvre.

				Au sein de la communauté il est possible de distinguer deux types d'acteurs:

				\begin{description}[font=\color{burntorange}]
					\item[les développeurs indépendants: ] Qu'il s'agisse de gloire, de monté en compétence sur un domaine ou d'altruisme, il existe des développeurs qui soutiennent le développement de produit et participent au support.
					\item[Les contributeurs et entreprises contributrices: ] Certaines entreprises favorisent l'aide et autorise leurs employés à travailler une partie de leurs temps d'activité sur des projets open sources.
				\end{description}

				Parmis ces contributeurs, on retrouve beaucoup de salariés d'entreprise IT et ce pour plusieurs raisons comme:

				\begin{itemize}[label=\textbullet, font=\LARGE \color{burntorange}]
					\item Le marketing: statuer que l'on a un développeur qui travaille sur un « Grand projet » pour dorer son image et promouvoir son entreprise.
					\item La gouvernance: car cela permet d'avoir son mot à dire sur les orientations stratégiques d'un produit
					\item Le socle technique: Plus il y a de contributions à un socle de produit open source dont l'entreprise est utilisatrice, meilleur sera leur business.
					\item La maitrise du produit: Monter en compétence et proposer du support sur ce produit.
				\end{itemize}

				\paragraph{Les contributions communautaires\\}

				%A retravailler
				Les éditeurs open source comptent en général assez peu sur les apports communautaires, du moins sur le coeur de leur produit. Ils les acceptent car c'est dans la logique de l'open source, mais ne les encouragent guère et l'on peut penser qu'il ne leur déplait pas de garder la maîtrise de leur produit.\\

				A noter que si son morceau de code est accepté, le contributeur devra généralement signer un accord spécifique qui permet à l'éditeur de disposer librement de son code. C'est assez naturel, car si chaque contributeur pouvait spécifier ses propres conditions de licence, le produit final serait un enchevêtrement de licences indémêlables.\\

				Afin de bénéficier d'une dynamique communautaire, tout en conservant la maîtrise du noyau de leur produit, certains éditeurs mettent en place un dispositif d'extensions, qui permet d'apporter des enrichissements au produit, de manière propre et indépendante du noyau, en assurant la compatibilité avec les versions futures.

				\paragraph{Le noyau et les extensions, un écosystème\\}

				Le modèle qui semble le plus efficace, et le meilleur compromis, est celui qui distingue le noyau du produit, sous la responsabilité de l'éditeur, et les extensions, réalisées par la communauté.

				Les principes de séparation sont les suivants:

				\begin{itemize}[label=\textbullet, font=\LARGE \color{burntorange}]
					\item Le noyau doit être d'une grande robustesse, il est certifié par l'éditeur, les contributions externes y sont rares
					\item L'inferface entre le noyau et les extensions est bien documentée et stabke, c'est à dire qu'un changement de version du noyau n'implique pas, du moins le plus souvent, un changement de version des extensions.	
					\item L'éditeur stimule la réalisation d'extensions, car elles donnent de la valeur à son produit et témoignent aussi de l'existence d'une communauté, en soi un gage de pérennité. L'éditeur offre en général une plateforme de mise à disposition de 
				\end{itemize}

				Ce modèle noyau/extensions est celui qui réalise le meilleur point d'équilibre entre les rôles respectifs de l'éditeur et de la communauté, réunissant la garantie et l'engagement de l'éditeur avec le dynamisme et l'énorme capacité de développement de la communauté.

			\subsubsection{Les supports de l'open source}

				% A retravailler
				Le support dans le monde du logiciel, c'est la capacité à apporter de l'aide dans l'utilisation du programme et à corriger le programme le cas échéant.\\
				
				Le support peut s'adresser aux utilisateurs finaux, comme aux exploitants du programme, ou encore aux programmeurs travaillant sur le programme.\\
				
				Le déploiement de programmes pour des tâches critiques, en particulier dans des entreprises, requiert absolument un support, car le risque d'une situation de blocage est trop important, cela que ce blocage soit dû à une anomalie ou à un mauvais usage, mauvaise configuration, incompatibilité, etc.

				\paragraph{Le support de l'éditeur\\}

				Du côté des éditeurs open source (MySql, eZ Publish, OpenERP, ...), la question est différente: l'éditeur est une société commerciale et son business model est essentiellement basé sur son offre de support. Ici donc, le dispositif de support est très proche de celui des produits propriétaires. Pas identique toutefois car \textit{en parallèle, en complément} au support payant de l'éditeur, il existe souvent un support communautaire, plus ou moins vivace selon les produits.Mais le plus souvent, les corrections touchant au code ne sont assurées que par l'éditeur.\\

				Pour les nouveaux éditeurs de l'open source commercial, le support produit est le fondement du business model, il est leur raison de vivre, leur unique source de revenus. On peut donc s'attendre à un support de grande qualité.

				\paragraph{Le support de la communauté\\}

				Les produits communautaires bénéficient avant tout d'un support communautaire. C'est à dire basé sur le volontariat de développeurs impliqués, qui répondent aux questions des utilisateurs sur les mailing-lists et forums. Et basé également sur le suivi et la prise en charge des anomalies sur les plateformes de développement communautaires.\\

				Lorsque la communauté est active, comme c'est le cas autour des grands produits, ce support communautaire peut être d'une très grande efficacité, d'une très grande réactivité, très supérieur à un support commercial.\\

				Nous verrons plus tard comment améliorer cette gestion de la communauté dans la partie optimisation des ressources.

		\subsection{Le consommateur}

			\subsubsection{Qui est le consommateur ?}

			Je classifie les consommateurs en trois catégories distinctes :
			\begin{description}[font=\color{burntorange}]
				\item[Les end-users:] Ce sont les clients finaux du produit open source. Si le logiciel open source à pour vocation d'être utilisé comme tel par le grand public ou bien parmis les entreprises, on considère ces personnes comme des utilisateurs finaux ou \textit{« end-users »}. Ils utilisent le logiciel, remontent leurs besoins d'amélioration, de correctifs et de support.
				\item[Les contributeurs et la communauté:] consomme l'open source car ils y contribuent, en adaptant certains besoins du logiciel par la création d'extension, par l'aide au développement, et tout soutient qui implique l'utilisation du produit de l'éditeur.
				\item[Les autres éditeurs et prestataires:] Comme nous l'avons aperçu précedemment, une licence open source permet de bénéficier de toutes les ressources sous la même licence. Les développeurs chez les éditeurs et prestataires réalisent des aggrégats de différentes solutions open sources à laquelle ils intégrent la leur. Nous le verrons plus en détail dans la partie concernant l'intégration de solution.
			\end{description}

			Il est important de considérer que nous somme tous des consommateurs appartenants à la catégorie « end-user » car aujourd'hui, nous avons tous utilisé au moins une application, une partie de logiciel qui utilise de l'open source. Même sans en avoir conscience.\\

			Favoriser l'open source pour ces consommateurs c'est leur permettre une expérience plus libre dans leurs besoins. Voyons à présent quels en sont leur bénéfices.
			
			\subsubsection{Les bénéfices de l'open source pour le client}
			%A retravailler

			Bien sûr, les bénéfices économiques sont parmi les premières raisons dans le choix de solutions open source. Même si « libre ne signifie pas gratuit », ces solutions ont toujours un coût de possession sensiblement moins élevé que leurs équivalents propriétaires.\\

			D'autant que les prix de prestations tendent aussi à être moins élevés, car l'ouverture du produit facilite la diffusion de la connaissance.\\

			Mais au fur et à mesure que ces solutions arrivent à maturité, le moindre coût n'est plus le premier critère de choix.\\
			Les principaux arguments sont alors :\\


			\begin{itemize}[label=\textbullet, font=\LARGE \color{burntorange}]
				\item \textbf{La non-dépendance}, ou moindre dépendance, par rapport à un éditeur. On sait que changer d'outil peut coûter très cher, et les éditeurs peuvent être tentés de profiter de la vache à lait que constituent ces clients devenus captifs. En anglais, on parle de vendor lock-in, le verrouillage par le fournisseur.
				\item \textbf{L'ouverture} est également un argument de poids. Les solutions open source sont en général plus respectueuses des standards, et plus ouvertes vers l'ajout de modules d'extension.
				\item \textbf{La pérennité} est un autre critère de choix fort, nous y revenons plus loin.
				\item \textbf{Et la qualité} finalement, car dans beaucoup de domaines les solutions open source sont réellement, objectivement, supérieures. Le très grand nombre de déploiements et donc de retours d'expérience, mais aussi leur modèle de développement et leur intégration de composants de haut niveau, permet à beaucoup de surclasser les produits propriétaires souvent vieillissants.
			\end{itemize}

			A quoi on peut ajouter le plaisir, pour les informaticiens, d'utiliser des programmes dont ils peuvent acquérir une totale maîtrise, sans barrière ni technique ni juridique.

		\subsection{Où trouver de l'open source ?} 

		Que l'on souhaite démarrer son projet open source ou contribuer au patrimoine du Logiciel Libre, de nombreuses plateforme et espace web permettent le stockage et la redistribution de ces projet open sources.
		
		\paragraph{Pour conclure\\}

			L’open source est donc un mouvement important. Il apporte des valeurs de liberté, solidarité mais aussi des bénéfices tant pour les citoyens que pour les entreprises. 
		
	% --------------------------------------------------%
	%     												%
	%             Optimisation des ressources			%
	%       										    %
	% --------------------------------------------------%
	\section{Optimisation des ressources} % in progress % Utiliser des livres sur comment monter et créer son logiciels open source et mettre en concordance led élement apporté dans Reflechissez et devenez riche + Oser la confiance

		\paragraph{Concept de l'optimisation\\}

	 		Le succès d'un projet open source dépend pour beaucoup de leur code, mais pas uniquement. De leur concept, de leur capacité à faire connaître votre projet, à faire naître une communauté, à piloter une startup aussi, et le cas échéant à convertir leur succès en revenus, qui pourront assurer la pérennité de leur aventure.

		\subsection{Rendre le libre populaire}

			\subsubsection{Pareto revisité}
			Une étude sur la participation de développeur et le code total rédigé rapporte que si l'on prend un projet ou 200 programmeurs participent, seulement, 10 d'entre eux ont écris 50\% du code.

			\subsubsection{L'enseignement du logiciel libre}

			Comme mentionné dans l'introduction de ce document, le logiciel libre me semble pas suffisemment enseigné. 

			\textbf{Qu'est-ce que j'entend par enseigner le logiciel libre ?}

			\paragraph{Susciter l'intérer du libre}

			Enseigner un logiciel libre par la mise en place d'efforts spécifiques à celui-ci. Ce n'est pas en utilisant sans le savoir de l'open source ou en cherchant spécifiquement à remplacer tout logiciel propriétaire par du libre que l'on enseigne l'open source. Il est nécessaire d'expliquer les mécanismes liés à l'open source, de prendre le temps d'informer sur toute l'importance et ce qui gravite autour de l'open source.\\
			Ceci afin de permettre à des centaines de programmeur éparpillé sur la planète à coopérer de façon cohérente sur la réalisation des millions de lignes de code.\\

			Celà passe également par la mise en relation des étudiants avec les communautés de développeur.

			\paragraph{Améliorer la recherche sur l'open source\\}

			Il est nécessaire d'encourager la recherche qui se développe autour des logiciels libre et fournir des outils nouveaux pour accompagner leur essor.\\

			Les plateformes distribuant l'open source sont à première vue trop réservée à la communauté de développeurs, des « Nerds » qui développent la nuit .

			\paragraph{Un gisement d'emplois futur\\}

			Tant par ses valeurs humanistes que par la contribution au patrimoine de l'humanité, l'open source doit être vue comme un bien commun qu'il faut cultiver en commun.\\

			Tout autant que l'art qui est exposé dans de nombreux musée que l'on rends accecibles les 1er dimanche du mois gratuitement pour contribuer à la culture de l'homme, l'open source devrait être considéré de même.\\

			Au coeur de l'activité industrielle encore méconnu, l'open source c'est \textbf{4,46 Md d' \euro{}} de \acrshort{ca} en 2017. \textbf{4000 emplois} nets ont été estimée d'ici 2020. La France est le leader Européen de l'open source. Et pourtant, le système éducatif actuel ne perçoit pas ce gisement. 

		\subsection{Intêret et communauté}

		\subsection{Le modèle noyau extension}

			Comme précedemment évoqué, il s'agit ici du meilleur schéma de développement connu pour déployer son activité open source. Il s'agit d'identifier et d'offrir noyau logiciel et mettre en place des points d'attaches pour les différentes extensions que l'on souhaiterai apporter.

			On dénote plusieurs avantages à ce modèle d'architecture:

			\begin{itemize}[label=\textbullet, font=\LARGE \color{burntorange}]
			\item Ne pas faire grossir inutilement un programme en se concentrant sur l'essentiel du business et de sa fonctionnalité.
			\item Jouer sur l'aspect modulable et simple du logiciel et concurrencer les logiciels propriétaires qui, sous pressions, rajoutent de plus en plus de fonctionnalités sur leur logiciel jusqu'à le rendre trop lourd .
			\item Déjouer la concurrence sur le marché, car la communauté se concentre sur la création et l' amélioration perpetuelle des extensions. Donnant un coup de neuf tout les jours.
			\item Eviter de destabiliser son logiciel car les fonctionnalités peuvent être ajouté sous forme de modules.
			\end{itemize}

			Le modèle noyaux-extension permet de tracer cette frontière entre l'éditeur et la communauté tout en s'assurant que la communauté trouve sa place au sein du produit et que chacun puisse répondre à son besoin.

			Pour favoriser la mise en place et le maintient du modèle noyau-extension, l'éditeur met généralement en place une plateforme pour accueillir ces extensions afin d'avoir une meilleur visibilité, classer par popularité, trier selon les besoins et de rendre gloire aux auteurs de ces extensions.
			
		\subsection{Eveiller sa communauté}

			Afin de faire naitre et grandir sa propre communauté autour de son produit open source, il existe déjà quelques points clés ou « best practices » dont il est bon de rappeler:

			\begin{itemize}[label=\textbullet, font=\LARGE \color{burntorange}]
				\item \textbf{Semer la graine} en mettant en ligne son code source. Être patient vis à vis de l'épanouissement de la communauté qui va prendre racine. Celà ne prendra pas quelques jours mais est un travail de longue halaine.
				\item Être \textbf{transparent} tant dans son projet et ses orientation que dans la \textbf{gouvernance} du projet.
				\item Choisir et \textbf{décréter son support d'échange} avec la communauté. Afin de centraliser la communauté sur ses outils.
				\item Instaurer sa \textbf{politique de l'open source} en proposant un modèle \textbf{noyau-extension}, celui qui fonctionne pour le mieux. Bien préciser quels sont les aspects que l'on veut retrouver dans notre \textbf{vision} open source.
				\item \textbf{Inspirer des valeurs}: le logiciel que l'on va construire n'a pas de but lucratif ou du moins il est moindre.Ceci aide à construire la communauté plus facilement car elle n'aura pas l'idée qu'on génère de l'argent sur son dos. On aspire à aider, changer le monde. Il faudra trouver l'idée d'\textbf{une mission} que les gens adoptent.
			\end{itemize}

	% --------------------------------------------------%
	%     												%
	%              Etude du consommateur				%
	%       										    %
	% --------------------------------------------------%
	\section{Etude du consommateur} % in progress

		\subsection{Les entreprises adoptent l'open source}

			L'open source croît sans cesse et il n'est plus une en,treprise dont une part du système d'information ne soit construit avec des composants et solutions open source. L'open source et partout on retrouve l'expression anglo-saxonne « Open source is going mainstream», l'open source est devenu un standard, une normalité.\\
			%A retravailler

			Pour une grande majoritée de clients, de DSI, quelle que soit la taille d'entreprise ou son scteur d'activité, les solution Open source sont désormais pleinement acceptées, et leur bénéfices parfaitement appréciés. C'est énorme. La banalisation ici, est bien une forme de consécration. Chez les prestataires, ce marché à attiré du monde notemment avec les \acrshort{ssii}.\\

			Le catalogue de service lié à l'open source s'aggrandit chaque jour et aide à convaincre les DSI les plus exigentes.

	% --------------------------------------------------%
	%     												%
	%            Le marché de l'open source			    %
	%       										    %
	% --------------------------------------------------%

	\section{Le marché de l'open source} % in progress

		\paragraph{Libre n'est pas gratuit \\} 

			C'est bien l'un des principes de l'open source. Même si l'on y a pour vocation de s'étendre et de distribuer son produit, il est nécessaire, si ce n'est vital pour certains éditeurs de trouver une source de revenus pour leur logiciel.\\
			Il faut savoir qu'il existe une différence entre faire payer un logiciel propriétaire et un logiciel open source. 
			\begin{description}[font=\color{burntorange}]
				\item [Payer un logiciel propriétaire: ] permet non seulement d'apporter des revenus à une entreprise mais d'obtenir un « droit de possession », d'acquisition du logiciel.
				\item [Payer un logiciel open source: ] N'est pas un prix d'acquisition ni un droit d'utilisation mais une source de revenus à l'éditeur pour permettre au logiciel de prospérer.
			\end{description}

		\subsection{Les acteurs de l'open source}
			Il existe dans l'open source 4 grands acteurs:

			\begin{description}[font=\color{burntorange}]
				\item[Les fondations:] telles que Apache ou Eclipse, sont des organismes à but non lucratif qui stimulent et pilotent le développment de grands produits open source.
				\item[Les distributeurs:] Redhat, Canonical (Ubuntu) ou Mandriva sont des distributeurs (très souvent éditeurs par la même occasion). Ils sélectionnent des outils et composants autour d'un noyau Linux, en assurent le packaging, la distribution et le support
				\item[Les éditeurs:] Diffusent des logiciels sous licence open source, ils réalisent la promotion de leurs produits et proposent du support
				\item[Les prestataires: ] Ils vendent des services sur l'open source. Il peut s'agir de conseil, d'intégration, de support, de la formation, des solutions d'hébergement, etc.
			\end{description}

			Je m'intéresse particulièrement aux éditeurs et aux prestataires de l'open source qui devront mettre en place des solutions pérénisant financièrement leur activité.\\

			Et de la prospérité les éditeurs en ont bien besoin car pour les autres acteurs, la taille, la mission et le produit présenté ne pose plus aucune difficulté de revenu. Doit-on encore se soucier du bon développement de Linux et de sa communauté ? Ces géants de l'open source, association à but non lucratif, ont des moyens marketings (plus de 60\% de leur revenus) et financier bien supérieurs aux éditeurs et prestataires qui restent pour la majorité des petites et moyennes entreprises.

			\textbf{Comment fonctionne donc le modèle économique ou « business model » de ces entreprises ?}\\

		\subsection{Chez les éditeurs}

			Même s'ils bénéficient de l'open source pour réduire leur coût de ressources, celle-ci ne comblent pas les besoins de financement de la partie développement en interne, hébergement, marketing. Il est donc nécessaire de trouver des sources de revenus pour les éditeurs.

			Parmis celles-ci, on distingue 3 principales sources pour l'éditeur:

			\begin{itemize}[label=\textbullet, font=\LARGE \color{burntorange}]
				\item Vendre des licences
				\item Vendre du support
				\item Vendre de l'intégration
				\item Autres sources de revenus
			\end{itemize}

			\subsubsection{Vendre des licences}

			Même s'il est interdit de faire payer l'utilisation d'un logiciel open source, les éditeurs ont bien compris commment faire bon usage de la « double licence ».

			\paragraph{Double licence commerciale\\}

			Il est possible de distribuer une oeuvre dérivée utilisant le programme sans diffuser ses sources à l'aide d'une licence commerciale.\\
			Le programme officiel open source est gratuit mais sa version dérivée elle est payante à travers l'achat de licence commerciale.\\

			Pour l'entreprise MySql, la vente de licence représente plus de la moitié du \acrfull{ca}. Pour d'autre, ce n'est pas une source de revenu suffisante car leur finalité est de faire un \textit{site} et non \textit{un produit}.

			\paragraph{Les extension payantes\\}

			L'éditeur peut proposer des extensions aux fonctionnalités présente dans le logiciel open source. Le logiciel initial est suffisemment de qualité et donne envie de payer quelques extension \textit{optionnelles} supplémentaire pour le confort et le besoin de l'utilisateur.

			\paragraph{Un support uniquement sur licence commerciale\\}

			Ici aussi, le logiciel est sous licence open source mais si l'on désire avoir le moindre support dessus, il faudra se tourner vers son confrère et sa licence commerciale.

			\subsubsection{Vendre du support}

			La prestation de support est une source principale de revenu pour l'éditeur, même s'il doit pour celà faire face à la concurrence potentielle de prestataires tiers que je précise juste après.

			Le support d'un logiciel inclut généralement :

			\begin{itemize}[label=\textbullet, font=\LARGE \color{burntorange}]
				\item L'accès privilégié aux correctif et ressources spécifique
				\item Une prise en chage des problèmes (anomalies, utilisation, mise en oeuvre ...)
				\item Des prestation d'audit, de certification, ou de prise de contrôle à distance, ainsi que la surveillance proactive et les corrections.
			\end{itemize}

			On peux distinguer différent modèles de financement concernant la vente de support:

			\paragraph{Uniquement le support\\}

			Certains éditeurs misent uniquement sur la vente de support. Le logiciel est gratuit mais le support lui est payant. Ce modèle de financement fonctionne pour certaines entreprises comme Tiny(OpenERP), Nuxeo.\\

			Le problème est qu'en cas ou le support n'a pas été utile l'année souscrite pour cause de stabilité du logiciel, le client voudra surement le résilier pour la suite.\\

			Plus le produit est de qualité, moins le support est facile à vendre car le client rencontre moins de difficultés.\\

			L'avantage est que plus on avance dans la technologie et plus la concurrence est présente on doit donc sortir des nouveautés constemment ce qui fragilise le produit et le rend instable.Le support est donc précieux dans le cadre professionnel.

			\paragraph{Faire payer la stabilité\\}

			Pour d'autre éditeurs, la stabilité d'un logiciel peut devenir source de profit. L'idée est de proposer deux logiciel:\\

			\begin{itemize}[label=\textbullet, font=\LARGE \color{burntorange}]
				\item Celui en licence open source sera classifié de « community edition ». Il sera présenté comme instable, pas entièrement testé, à ne pas déployer en production
				\item Tandis que le logiciel sous licence non-libre sera la licence sécurisée, une version « enterprise-ready », « fully tested ».
			\end{itemize} 

			Au final, la version community, c'est celle en cours de développement donc en avance de phase alors que la version entreprise c'est celle qui a été \textit{gelée} dans un état dit \textit{« stable »}.\\

			L'éditeur peut alors diffuser et promouvoir son produit à travers la version community, et apâter les entreprises dans la version payante ou généralement le support est packagé avec.\\

			L'éditeur jongle sur la stabilité de la version community en la rendant suffisamment stable pour donner envie de l'utiliser mais inciter fortement les entreprises à prendre la version sous licence commerciale qui sera accompagnée du support.

			\paragraph{Fonctionnalités avancées\\}

			Pour l'éditeur, un autre business model et celui de la fonctionnalitée avancée.\\
			La version community et enterprise n'est pas différente d'un point de vue stabilité mais ce sont les fonctionnalitées présente qui sont réduites sur la version community.\\

			Ceci permet de se dégager du paradigme « libre = instable » totalement infondé.\\

			La difficulté pour l'éditeur va être d'avoir suffisemment de fonctionnalitées pour rendre la version libre intéressante mais d'avoir une forte valeur ajoutée sur les fonctionnalités dans la version payante.

			\subsubsection{Vendre de l'intégration}

			Pour les éditeurs qui ne sont pas mondialement connu, il est possible de trouver la rentabilité en proposant l'intégration de son produit open source. C'est un moyen de démarrer dans le milieu sans trop de risque mais qui n'est pas « \gls{scalable} ». On ne peux s'étendre à l'étranger si l'on est le concurrent direct de ses intégrateurs partenaires. On reste donc sur un marché réduit.

			\subsubsection{Autres sources de revenus}

			\paragraph{Les campagnes de crowdfunding\\}

				En Mai 2019, la plateforme d'hébergement de logiciel Github, que nous détaillerons plus tard, propose un programme de sponsoring et de crowdfunding.

			\paragraph{Vendre des extensions\\}

				Selon le modèle noyau-extension, nombre d'éditeur open source on trouvé leur part de marché dans la vente des extensions. Bien que réalisées généralement par la communauté, l'éditeur peut considérer la vente des extensions réalisées afin de rémunérer les contributeurs mais prélève un montant sur la vente d'extensions. De cette manière, l'éditeur de la solution open source Magento pour le e-commerce, prélève 30\% du prix de vente des extensions à la manière de l'Apple Store.
			
		\subsection{Chez les prestataires}

			Pour les prestataires de l'open source, le business model est basé sur la vente de prestation et d'expertise sur les produit.

				\paragraph{La naissance des \acrfull{ssll}\\}

				L'une des activité du prestataire est la gestion du support de solution open source. Il existe une telle diversité de composant open source qu'il est difficile pour un prestataire unique de se concentrer sur le support dudit logiciel.\\

				Dans certains cas, la prestation est demandé par l'éditeur. Dans d'autre, c'est un prestataire tiers qui assurera du support, sans doute moins légitime et moins expert que l'éditeur lui-même mais aussi moins chère.\\

				Ainsi est né en France le concept de \acrshort{ssll}, une société qui propose d'assurer le déploiement et le support de configuration multi-produits à base d'open source.\\

				Un peu plus tard les \acrfull{ssii} généralistes prennent partis en ouvrant des centre de support open source.

				\paragraph{Intégrer des solutions open sources\\}

				Le principe de ce modèle est de construire une application globale, un système d'information, en assemblant des logiciels open sources.\\

				Il faudra donc également assurer le support de ces créations.\\
				La valeur ajoutée ici est dans le choix des solutions à intégrer. 

				Il y a tellement de possibilité de solution dans l'open source qu'il est nécessaire d'étudier correctement le marché en fesant les bons choix. Ceci passe par une action de veille permanente sur les solutions disponible, afin de déceler les opportunité et produits prometteur pour avoir la création la plus solide et pérenne possible.

		\subsection{Communication par ses atouts}

			En guise de communication, l'open source n'a pas tant besoin de budget marketing. En effet il puise sa force là ou il a pris racine, c'eest à dire dans sa communauté. Nul besoin de mettre de faux posts et avis sur touts les blogs du net, la vérité et la promotion de l'open source se fait principalement par la communauté.

			Le caractère open source permet en général de diffuser bien plus rapidement son produit.

			La communauté utilise donc tout support moderne de communication pour diffuser, twitter, poster ces informations

			Ainsi malgré le marketing ordinaire (campagne publicitaires, affiches, buzz-marketing ...) que ne peuvent se permettrent bon nombre d'éditeur, ils peuvent espérer que le marketing moderne 2.0, qui est la force et le fondement de l'open source, prenne la relève.












\chapter{Etude terrain} % 30 pages

	Lors de mon étude terrain, j'ai eu l'occasion de recenser l'avis de nombreuses personnes sur l'open source mais également sur les réflexions menée lors de mon état de l'art.

	Pour cette étude terrain j'ai pu réaliser un sondage de 13 questions autour de ma problématique qui à pu être traité par 38 personnes qui sont lié de près ou de loin à l'open source. Cette étude quantitative me permet de vérifier la véracité de mes hypothèses en cherchant le maximum d'approbation mais aussi de désapprobation de mes idées et réflexions.

	Egalement, l'étude qualitative que j'ai pu réalisé au travers de 4 interviews, m'a permis de confronter mes idées à celles d'autre personnes sensibles au domaine de l'open source . Ceci m'aide à étayer mes réflexions à travers leurs visions.

	J'ai choisi d'orienter mes questionnaires et interviews sur les 4 grands domaines qui représentent selon moi les piliers à batir pour valoriser l'open source, et sur lesquel l'éditeur à la main.

	\section{Plateforme promotrice}

		\subsection{Une interface pour communiquer qui laisse à désirer}

			Autour des plateformes qui contiennent et promouvoie l'open source, sur 31 personnes enregistrées pour cette question, 20 indiquent qu'il y a un manque à palier dans l'interface qui permet de communiquer avec l'éditeur open source. 

			\begin{figure}[ht]
				\center
				\includegraphics[scale=0.28]{./img/a9}
				\caption{Communication avec l'éditeur}
			\end{figure}

			Il apparait donc que \textbf{la communication qui a un aspect fondamental} pour l'open source \textbf{peux clairement être améliorée} afin de satisfaire non seulement les besoin dans la communication auprès de l'éditeur mais surtout le besoin du consommateur à communiquer correctement.

			Lors de l'interview auprès de Olivier \bsc{Magnial}, ingénieur systèmes embarqués chez l'une des plus grande entreprise promotrice de l'open source : SMILE. Celui-ci à déclaré : 

			\begin{center}
				\textit{
				\textquote{
					Pour \gls{mainliner} du code source, un processus décrit la manière de contribuer, et c'est le plus souvent par mail.(...) Linux, par exemple c'est entièrement du mail, on à des mailing list extrêmement longues et des processus assez carrés !
				}
				}
			\end{center}

			Ainsi même si un moyen de communication et un processus est bien présent pour la contribution. \textbf{Ce n'est pas au travers des plateformes que cette communication pour la contribution est la plus utile}.

			De plus la prise en main d'un logiciel open source est souvent plus compliqué nous révèle Quentin \bsc{Cazelle}, ingénieur logiciel chez Docdoku.\\

			Pourquoi selon lui ?

			\begin{center}
				\textit{
				\textquote{
					Car il y a des fonctionnalités non documentées (...) les plateformes sont incompletes car les développeurs qui contribuent au projets open source ne s'embetent pas à la documentation et a bien expliquer les \gls{issues}
				}
				}
			\end{center}

			J'en déduis donc qu'en plus d'une communication pouvant être amélioré, \textbf{faciliter l'écriture et sensibiliser les consommateurs à l'édition de la documentation} est un axe d'amélioration potentiel.

		\newpage

		\subsection{Un module de présentation}

			Sur la question:

			\begin{center}
				\textit{
				\textquote{
					Les plateformes pour communiquer, contribuer à l'open source vous conviennent-elles ?
				}
				}
			\end{center}

			Seulement 8 personnes sur 31 personnes qui ont répondu à celle-ci sont intéressés pour avoir une présentation sous forme de vidéo du projet, avec un mot de l'éditeur qui communique ses ambitions, ses valeurs.

			\begin{figure}[ht]
				\center
				\includegraphics[scale=0.28]{./img/a92}
				\caption{Module de présentation}
			\end{figure}

			Lors de l'interview de Quentin \bsc{Cazelle}, Ingénieur développeur chez Docdoku, celui-ci mentionne tout de même le fait qu'une vitrine à ces plateformes s'impose pour les consommateurs finaux qui ne sont pas développeur.

			\begin{center}
				\textit{
				\textquote{
					La plateforme est un frein pour l'utilisateur final non développeur, il faudrait en effet mettre une vitrine dans un style plus commercial (...)
				}
				}
			\end{center}

			J'en conclus qu'\textbf{une vitrine de présentation du logiciel est bel est bien intéressante} mais \textbf{d'un point de vue utilisateur final} et non développeur contributeur. Ma question dans l'interview ne mentionnais pas la cible de ce module de présentation d'ou le nombres de réponse tout de même favorables à ce module.

		\subsection{Pas d'extrèmes sur les plateformes}

			Très peu de contributeur, c'est-à-dire 6 sur 30, répondent que les plateformes sont parfaites et qu'ils ne voient pas d'amélioration potentielles.

			Il n'y a pas non plus beaucoup d'insatisfait sur celles-ci car seulement 3 ont répondu que toute l'interface était à refaire.

			Je trouve donc que \textbf{les plateformes promotrices sont utilisés et essentielles} à l'open source et ne sont pas à évincer pour l'éditeur.

		\newpage

	\section{Gestion des ressources}

		\subsection{Un mot de l'éditeur pour valoriser la contribution}



		\newpage
	\section{Chez le consommateur}

		\subsection{La contribution du consommateur}

			Dans les personnes interrogés, beaucoup souhaite contribuer ou ont déjà contribué à l'open sources ou souhaitent le faire un jour prochain.60\% y ont déjà contribué, 28\% souhaitent y contribuer un jour.

			\begin{figure}[ht]
				\center
				\includegraphics[scale=0.58]{./img/a4}
				\caption{Contribution à l'open source}
			\end{figure}

			L'open source est donc un sujet qui les intéresses et dont \textbf{ils peuvent ou veulent investir du temps en y contribuant}.

			Et ce quelque soit leur domaines d'activité. Sur les 37 personnes qui ont répondu, les domaines d'activités, même si une majoritée est dans l'informatique, sont divers:

			\begin{itemize}[label=\textbullet, font=\LARGE \color{burntorange}]
				\item Edition, Communication, Multimédia
				\item Etude et conseils
				\item Informatique / Télécom
				\item Industriel
				\item Autres
			\end{itemize}

			\begin{figure}[ht]
				\center
				\includegraphics[scale=0.58]{./img/a1}
				\caption{Secteur d'activité des personnes interrogées}
			\end{figure}

			Ainsi \textbf{l'open source n'est pas seulement présent dans l'informatique} et est une préoccupation pour les personnes interrogés.

		\subsection{Le ressenti du consommateur à contribuer}

			J'ai posé une question dans mon questionnaire autour de la perçeption que les gens peuvent avoir dans l'accueil de contribution, s'ils avait des peur ou au contraire qu'il est très agréable de partager avec l'editeur et la communauté ses contributions.

			Globalement, aucun frein n'est ressenti à la contribution et son accueil par l'éditeur, même si ces personnes ne contribuent par pour autant:

			\begin{itemize}[label=\textbullet, font=\LARGE \color{burntorange}]
				\item 46\% des interrogés ont répondu que leurs contribution était très bien accueilli, que l'éditeur et la communauté était agréable.
				\item 43\% Ne prennent pas le temps de contribuer mais n'y vois aucun blocage.
				\item Et seulement 11\% ont peur de contribuer et d'être jugé.
			\end{itemize}

			J'en déduis que la moitiée des personnes ont besoin de \textbf{plus de sensibilisation sur l'importance et les enjeux de la contribution} pour augmenter le nombre de contribution globale.

			Dans leur entreprise, ces personnes considèrent pourtant majoritairement que l'open source est essentiel ou nécessaire.

			Pour 10 personnes, l'open source est essentiel et ils y attachent beaucoup d'importance.
			12 questionnés disent que l'open source est nécessaire dans leur projets. 8 personnes disent que leur entreprise ne s'en soucie pas vraiment et 5 personnes n'ont jamais entendu parlé d'open source dans leur société.

			Ainsi malgré le degré d'importance de l'open source les personnes interrogés n'en font pas une affaire personnelle.

			\begin{figure}[ht]
				\center
				\includegraphics[scale=0.58]{./img/a7}
				\caption{Perception de contribution à l'open source}					
			\end{figure}

			\subsection{Le support payant}

			Dans l'ensemble des personnes interrogés, une forte majorité indiquent qu'ils ne sont pas contre payer du support pour un logiciel open source. 12 personnes ont répondu qu'il était nécessaire et abordable en général. 22 personnes n'ont pas d'objection à payer pour du support logiciel et comprennent qu'il faille rémunérer l'éditeur d'une certaine façon. Et seulement 3 ont répondu qu'il pensait que l'open source devait être gratuit

			\begin{figure}[ht]
				\center
				\includegraphics[scale=0.58]{./img/a11}
				\caption{Payer du support logiciel}					
			\end{figure}

			C'est donc que le modèle économique de l'éditeur à travers \textbf{la vente de support logiciel n'est pas un frein à la consommation de l'open source} car beaucoup de personnes interrogées estiment cela comme une moindre chose pour les éditeurs.

		\subsection{Un besoin écouté}

		29 des 37 personnes interrogés ont déclarés qu'après une demande auprès d'un éditeur open source, le besoin du consommateur est suffisemment écouté et ce malgré le fait que les plateformes ne favorisent pas cette communication.

			\begin{figure}[ht]
				\center
				\includegraphics[scale=0.28]{./img/a12}
				\caption{Ecoute du besoin du consommateur}
			\end{figure}

		Ainsi \textbf{la manque de communication dans l'expression du besoin dans l'open source relève d'un problème technique et matériel} plus qu'humain.

	\newpage

	\section{Marketing de l'open source}

		\subsection{L'école et l'open source}

			Sur une 30aine de personnes qui ont répondu à la question: "Devrait on sensibiliser les gens à l'open source dans les écoles informatiques ?", je constate que 12 de ces personnes ont découvert l'open source par le biais de l'école.

			\begin{figure}[ht]
				\center
				\includegraphics[scale=0.28]{./img/a3}
				\caption{Découverte de l'open source}
			\end{figure}

			Une majoritée des contributeurs, soit 92\% mentionne que l'on devrait bel et bien sensibiliser les gens à l'open source dans les écoles informatiques

			Ceci m'indique que l'enseignement de l'open source dans les école leur à été favorable et qu'ils recommandent donc que le programme contienne un enseignement à l'open source

			Seulement 3 personnes, dont 2 qui ont découvert l'open source à l'école, trouvent que c'est déjà fait intrinsèquement au programme.

			Ainsi \textbf{l'open source est bien un sujet important à traiter au sein de l'école.}

			\begin{figure}[ht]
				\center
				\includegraphics[scale=0.28]{./img/a6}
				\caption{Sensibiliser à l'open source}
			\end{figure}

			A ceci, Quentin \bsc{Cazelle}, qui sort d'une école d'informatique m'indique que son école traitais bien de l'open source et que le sujet était bien présent:

			\begin{center}
				\textit{
				\textquote{
					A l'IUT, j'ai eu des cours sur l'open source, c'était très léger mais on a compris le concept (...) c'est le minimum mais aussi peut-être le maximum que l'on puisse traiter sur ce sujet mais c'est nécessaire. 
				}
				}
			\end{center}

			Il ajoute à cela que l'open source est une brique nécessaire pour le métier de développeur logiciel.

			\begin{center}
				\textit{
				\textquote{
					Quelqu'un qui fait 5 ans d'étude de développeur et ne sait pas ce qu'est l'open source, c'est une abération!
				}
				}
			\end{center}

			Egalement, plus de la moitiée des personnes, soit 21 interrogés ont répondus que l'open source était peu ou tout juste assez évoqué à l'école.

			9 personnes trouvent que les écoles informatiques traitent suffisamment de l'open souce

			Seulement 1 personne à déclaré que l'open source était fortement présent dans les écoles informatiques

			\begin{figure}[ht]
				\center
				\includegraphics[scale=0.28]{./img/a5}
				\caption{L'open source à l'école}					
			\end{figure}

			J'en conclus que \textbf{ce n'est pas systématique mais l'open source est traité dans les écoles d'informatiques et il se doit de l'être} compte tenu de l'importance qu'il joue pour l'avenir des jeunes diplomés dans leur futures entreprises.










\chapter{Confrontation} 

\section{Promotion de l'open source}

	Ma première hypothèse concerne le manque de promotions, de mise en avant des produits open sources, et du manque d'outils disponibles sur les plateformes d'hébergement de code open source pour en faire le produit phare de tout les développeurs et entreprises du logiciel.

	\subsection{Des plateformes améliorables tout de même}

		Dans mon étude autour de l'open source, je me suis aperçu de la multitude des plateformes disponibles pour présenter les projets open sources et permettre la contribution. Malgré cela, il me semblait nécessaire de mettre en place une interface pour accueillir les potentiels futurs contributeurs, une vitrine permettant de visualiser le produit.\\

		Avec l'étude terrain et les résultats à mes questions autour de ce sujet, je me suis aperçu que les plateformes n'ont pas nécessité à promouvoir le projet car l'aspiration à la contribution n'est pas lié à la motivation et au marketing qui gravite autour. La contribution est basé sur le besoin technique qu'a le consommateur à utiliser une spécificité du logiciel open source et donc à adapter le produit à celui-ci.\\

		Néanmoins il s'en est dégagé un besoin crucial d'améliorer la communication entre toutes les parties prenantes du projet open source.

	\subsection{Le marketing de l'open source}

		Au cours de mon étude, j'ai découvert la promotion inhérente à l'open source dans les livres blancs mis à disposition par SMILE, j'ai eu tendance à penser que le marketing 3.0 par l'ouverture du code et donc le partage de celui-ci suffisait.\\

		Dans l'étude terrain et notamment lors de mon interview avec un éditeur open source, je me suis rendu compte qu'il est tout de même nécessaire de réaliser un marketing classique autour du produit à promouvoir afin de se faire une place sur le marché.

\paragraph{Réponse à l'hypothèse\\}

	Mon hypothèse concernant la nécessité de mettre en place une promotion à l'aide de vitrine sur les plateforme n'est pas valide, toutefois, un marketing pour promouvoir son produit peut etre réalisé en dehors pour attirer les entreprises et consommateurs.

\section{Optimisation des ressources}

	\subsection{Business model de l'open source}

		La meilleure gestion de projet open source que j'ai pu déniché s'apparentait au modèle noyau / extensions qui prévalait les conflits entre l'éditeur et les besoins communautaires. Egalement j'ai pu découvrir le management participatif au travers d'ouvrages sur le sujet que je souhaitais mettre en avant dans la gestion de projets open sources.

		En confrontant mon hypothèse à la vision des personnes interrogées et interviewées, je m'aperçoit que l'aspect hiérarchique horizontal en ressort bel et bien ce qui confirme mon hypothèse sur la gestion idéale des projets open sources. \\

		Néanmoins, je garde à l'esprit qu'il n'y a pas vraiment de business models applicable généralement car les besoins du projet open source varient constamment.

	\subsection{Gestion des ressources humaines}

		Dans les différents ouvrages et mes recherches effectuées, je souhaitais profiter de l'aspect communautaire avec de nombreuses ressources humaines, pour les organiser à travers l'utilisation d'un cerveau collectif et donc ajouter un "semblant" de gestion d'équipe et de projet.\\

		Ceci étant, avec les différents entretiens et échanges je m'aperçois que ma réflexion semble erronnée.
		En effet j'en ai conclus que l'open source se doit de se faire par le bon vouloir des personnes qui est souvent guidé par leurs besoins de développer un module en particulier.\\

		Ainsi l'éditeur ne peut avoir aucune mainmise sur les contributeurs et la gestion de leurs participations au projet.
		Il apparait néanmoins que la mise en place de guides de contribution, la facilitation des premières contributions et l'amélioration de la communication sont des réponses acceptable à ce besoin d'optimisation. 

\paragraph{Réponse à l'hypothèse\\}
	
	Je valide donc à 70\% mon hypothèse sur l'optimisation des ressources disponibles car il est tout de même possible d'améliorer le rapport humain afin de motiver et faciliter la contribution.

\section{Envies et besoins de contribuer}

	Ma troisième hypothèse traite du manque de sensibilisations qui plane sur l'open source, faisant de celui-ci un sujet de malentendu, d'incompréhensions par les consommateurs potentiels. Si l'on sensibilisait plus le consommateur de l'open source sur l'importance de leur contributions, alors le blason de l'open source en serait redoré et attirerait encore plus de monde.

	\subsection{Sensibiliser le grand public}

		Lors de mes recherches autour de l'open source et en comparant avec mon vécu dans mes entreprises et à l'école, je me suis aperçu du manque de sensibilisation sur le vaste sujet qu'est l'open source.\\

		Je confirme mon hypothèse sur le fait que l'on entends pas parler de l'open source pour le grand public car c'est le meme ressenti qui est partagé par les personnes interviewé et ayant répondu au questionnaire.\\

		Pour autant lors de mes différents échanges, je m'aperçoit que la cible du grand public est erronée. En effet, l'open source est généralement à destination des développeurs, des entreprises et dans ce monde là, on est normalement sensibilisé à l'open source.\\

		C'est une erreure d'interprétation que j'ai eu en faisant l'amalgame du consommateur et de l'utilisateur final du produit. Les utilisateurs finaux (ou grand public) ne sont pas forcément développeurs, et les produits open source ne sont pas tous des logiciels à destination de ces utilisateurs.\\

		Ainsi s'il n'est pas nécessaire de sensibiliser le grand public, il n'en est pas de même pour les entreprises.

	\subsection{Sensibiliser l'entreprise et le contributeur}

		En entreprise, le sujet de l'open source est considéré comme très important, on en entend parler, on l'utilise même beaucoup et c'est ce que j'ai pu constater également de mon coté mais aussi par les échanges que j'ai dans mes interviews.\\

		Les personnes concernés ont acquiescés le fait que l'entreprise à besoin de l'open source, qu'elle s'attend à ce que les développeur s'y connaissent sur le sujet... Pour autant, ils ne souhaitent pas forcément apporter leur pierre à l'édifice.\\

		Aujourd'hui, il est donc important de sensibiliser les entreprises et de trouver le moyen de les faire contribuer au monde du logiciel ouvert.\\

		En effet, la contribution de l'entreprise est un axe exponentiel de croissance pour l'open source.\\

		Une entreprise qui contribue en ouvrant son code, sera sensible aux autres entreprises qui font de même. Il sera alors plus facile pour elles de permettre à leurs développeur de contribuer à l'open source tant pour améliorer leur propres code en interne que pour utiliser celui des autres et l'adapter à ses besoins.\\

		Le développeur contributeur sera donc sensibilisé à son tour.

	\subsection{L'expression du besoin}

		Dans mes recherches, il ressortait un manque d'expression du besoin du consommateur et un manque de compréhension de la part de l'éditeur.\\

		Après l'étude terrain, je me suis aperçu que les consommateur ne se plaignent pas de l'écoute de l'éditeur concernant leur besoins mais néanmoins il apparait certains blocages à l'open source comme le manque de documentations.\\

		Ainsi les besoins sont exprimés mais la communication dans les projet open source est faible.

\paragraph{Réponse à l'hypothèse\\}

		Sensibiliser à l'open source les entreprises et offrir l'opportunité au consommateur de communiquer autour de ses besoins est donc une hypothèse valide à 90\%. Néanmoins l'utilisateur final, non développeur n'a pas véléité à être sensibilisé au sujet.











\chapter{Transposition} % 3/4 pages

Afin de répondre à ma problématique suivante:

\textbf{Comment valoriser, en tant qu'éditeur, l'open source et en faire la solution privilégiée des consommateurs ?}

Je vous apporte donc des préconisations à mettre en place autour de différents domaines.

\section{Améliorer la communication}

	Autour de ce travail, j'ai pu dégager l'évidence d'un besoin de communication vital pour le bien être du consommateur de l'open source mais également de l'éditeur, ainsi je préconise aux éditeurs de logiciel open source d'accentuer leur documentations pour une meilleure communication.

	\paragraph{Documentation inhérente au projet\\}

	Autour de chaque projets hébérgés sur les plateformes, une page d'accueil est généralement disponible. Le "ReadMe" est le fichier qui est présenté dans cette page et c'est celui qui doit être le plus complet pour appeler à la contribution.\\

	Il doit ainsi contenir les informations essentielles pour la contribution. Comment s'y prendre, quelles sont les étapes pour cela, les prérequis, les étapes de validations tout en restant le plus simple possible. Il existe même une méthode de développement autour de ce ReadMe, le \emph{Readme Driven Development}, consistant à mettre la priorité sur le Readme avant tout développement.

	\paragraph{Utilisation d'outils de communication\\}

	Pour développer en équipe, il existe de nombreux outils facilitant la communication. Mettre à disposition un espace de partages et d'échanges comme le propose Slack ou Reddit permettra de gérer les communications au travers de la communauté et de faire ressortir les idées et besoins.

	\paragraph{Intelligence collective\\}

	Cyber connecter l'intelligence collective des entreprises, des développeurs afin de bâtir les briques logicielles open source de demain. Avec la mise à disposition d'une interface non pas pour contribuer mais pour discuter et échanger sur les besoins de demain et les projets à batir.

\section{Sensibiliser à l'open source}

	Dans le but de promouvoir son propre produit open source, il peut être intéressant de sensibiliser les entreprises et développeur.

	\paragraph{Conférence sur l'intéret économique de l'open source\\}

	Participer aux différents évènements présents autour du monde du logiciel en apportant un témoignage de votre expérience de l'open source afin de solliciter les entreprises à contribuer en open sourçant leur produits également.

	Ceci apportera plus de consommateurs à l'open source et de contributions inter-entreprises.

	\paragraph{Accompagner à l'open source\\}

	Il peut être envisagé de se faire accompagner quand l'on souhaite open sourcer son code et continuer sur ce modèle, certaines connaissances et mise en place sont nécessaires. Ainsi proposer des services d'accompagnement aux futurs éditeurs permettra de croître le nombre de produits open sources sur le marché.

\section{Faciliter la contribution}
	
	\paragraph{Accompagner le contributeur\\}

	Le développeur qui souhaite contribuer à un projet open source se doit d'être accompagné dans sa démarche lors de ses premières contributions. Un accompagnement personnalisé est un plus qui peut amener facilement de nouveaux consommateurs à votre produit.\\

	La mise en place de didacticiels, d'échanges avec le contributeur permet d'améliorer la qualité des relations humaines dans le projet mais également de s'assurer de la bonne contribution du volontaire.

	\paragraph{Un guide de contribution\\}

	Pour faciliter la tâche à l'éditeur, un guide de contribution sur un format clair et rapide à lire est un atout essentiel.
	Il permet non seulement d'attirer la contribution mais également de gagner du temps d'échange, d'instruction à l'éditeur envers la communauté.\\

	Se référer à celui-ci permettra d'avoir un exemple de ligne de conduites à suivre pour participer à l'open source sans difficultés.
\chapter{Conclusion}

\paragraph{La thèse professionnelle\\}

	L'open source est un vaste océan qui habite le monde de l'informatique. Très utilisé et pourtant peu enseigné, il est intéressant de s'en approcher pour en comprendre les fondements essentiels. 

	Cette thèse professionnelle m'a permis non seulement d'en avoir une vision claire et précise mais également d'en comprendre les limites et les possibilités qui gravitent autour.

	L'intérêt que l'on peut avoir a y contribuer ou tout simplement mieux choisir ses futures briques logicielles qui feront parti intégrante de notre travail.

	Ne pas réinventer la roue, offrir à chacun la possibilité de répondre spécifiquement à son besoin et le faire partager, ce sont les valeur inhérente à l'open source

	J'ai souhaité à travers ma problématique, y apporter un peu de moi à travers mes réflexions personnelles sur l'utilisation de l'intelligence collective et les alternatives au management habituel.Même si cette gestion fine d'un projet open source n'est pas envisageable compte tenu de la liberté qui en découle, il apparait tout de même que l'humain et ses besoins technologiques est au coeur de la raison de l'open source et qu'il est donc primordial de mettre la communication en première ligne dans un projet open source.

\paragraph{Mon avenir\\}

	Dans un futur plus ou moins proche, je désire parvenir au poste d'architecte logiciel.\\
	En effet, lors de mes précédentes années de travail, j'ai pu rencontrer et travailler aux cotés de personnes sensibles à l'architecture logicielle qui m'ont transmise le goût de la réflexion et de l'investissement inhérents à ce métier.

	Pour aspirer à ce futur, la route est longue et j'ai donc réalisé ma roadmap vers mon métier cible.\\
	Premièrement récupérer les briques dont je n'ai pas eu la chance de découvrir dans mes années de travail et d'écoles comme l'algorithmie.
	Ensuite développer mes connaissances autour du développement logiciel, architecture serveur, gestion de projet, sécurité.
	Egalement, monter en compétences sur les différents blocs de communication nécessaire pour le métier d'architecte logiciel.

	J'attend donc avec impatience la fin (toute proche) de ces années d'études pour me plonger dans le coeur de cette aventure vers l'architecture logiciel, aventure dont vous entendrez peut être conter le récit.
\begin{appendices}

\chapter{Contexte de l'open source}

\begin{figure}[h]
	\center
	\includegraphics[scale=0.65]{./img/Domain_os.png}
	\caption{Domaine d'application de l'open source dans l'IT}							
	\caption*{\color{silver}Source: redhat.com}					
\end{figure}

\begin{figure}[h]
	\center
	\includegraphics[scale=0.65]{./img/Use_os.png}
	\caption{Secteur d'application de l'open source}
	\caption*{\color{silver}Source: redhat.com}					
\end{figure}

\end{appendices}

\end{document}